% Options for packages loaded elsewhere
\PassOptionsToPackage{unicode}{hyperref}
\PassOptionsToPackage{hyphens}{url}
\documentclass[
  12pt,
  openany]{book}
\usepackage{xcolor}
\usepackage{amsmath,amssymb}
\setcounter{secnumdepth}{5}
\usepackage{iftex}
\ifPDFTeX
  \usepackage[T1]{fontenc}
  \usepackage[utf8]{inputenc}
  \usepackage{textcomp} % provide euro and other symbols
\else % if luatex or xetex
  \usepackage{unicode-math} % this also loads fontspec
  \defaultfontfeatures{Scale=MatchLowercase}
  \defaultfontfeatures[\rmfamily]{Ligatures=TeX,Scale=1}
\fi
\usepackage{lmodern}
\ifPDFTeX\else
  % xetex/luatex font selection
  \setmainfont[]{Times New Roman}
\fi
% Use upquote if available, for straight quotes in verbatim environments
\IfFileExists{upquote.sty}{\usepackage{upquote}}{}
\IfFileExists{microtype.sty}{% use microtype if available
  \usepackage[]{microtype}
  \UseMicrotypeSet[protrusion]{basicmath} % disable protrusion for tt fonts
}{}
\makeatletter
\@ifundefined{KOMAClassName}{% if non-KOMA class
  \IfFileExists{parskip.sty}{%
    \usepackage{parskip}
  }{% else
    \setlength{\parindent}{0pt}
    \setlength{\parskip}{6pt plus 2pt minus 1pt}}
}{% if KOMA class
  \KOMAoptions{parskip=half}}
\makeatother
\usepackage{longtable,booktabs,array}
\usepackage{calc} % for calculating minipage widths
% Correct order of tables after \paragraph or \subparagraph
\usepackage{etoolbox}
\makeatletter
\patchcmd\longtable{\par}{\if@noskipsec\mbox{}\fi\par}{}{}
\makeatother
% Allow footnotes in longtable head/foot
\IfFileExists{footnotehyper.sty}{\usepackage{footnotehyper}}{\usepackage{footnote}}
\makesavenoteenv{longtable}
\setlength{\emergencystretch}{3em} % prevent overfull lines
\providecommand{\tightlist}{%
  \setlength{\itemsep}{0pt}\setlength{\parskip}{0pt}}
\usepackage[]{natbib}
\bibliographystyle{apalike}
\usepackage{booktabs}
\usepackage[top=1cm,bottom=1cm,left=1.5cm,right=1.5cm,includehead,a4paper]{geometry}
\usepackage{fancyhdr}
  \pagestyle{fancy}
  \fancypagestyle{plain}{%
  \renewcommand{\headrulewidth}{0pt}%
  \fancyhf{}%
}
  \fancyhf{}
  \fancyhf[HRO,HLE]{\thepage}
  \fancyhf[HC]{\leftmark}

%% Book Structure
\usepackage{booktabs}
\usepackage{titlesec}
\usepackage{caption}
\usepackage{subcaption}

\usepackage[no-math]{fontspec}
\usepackage{lmodern}
\renewcommand{\familydefault}{lmss}
\usepackage{amsthm}
\usepackage{mathtools}
\usepackage{unicode-math}

%% Utilities
\usepackage{tcolorbox}
\usepackage{hyperref}
\usepackage{tabularx}
\usepackage{multirow}
\usepackage{array}
\usepackage{wrapfig}
\graphicspath{ {images/} }
\usepackage{exsheets}
\usepackage{setspace}
\usepackage{etoolbox}
\usepackage{enumitem}
\usepackage{graphbox,graphicx}
\usepackage{float}
\usepackage{adjustbox}
\usepackage{quotchap}
\usepackage{dirtytalk}
\usepackage{setspace}
\usepackage{multicol}
\usepackage{epigraph}
\usepackage[switch, modulo]{lineno}
\usepackage{vwcol}

%% Maths and Tikz
\usepackage{tikz}
\usepackage{tikzlings}
\usetikzlibrary{math}
\usetikzlibrary{patterns}
\usetikzlibrary{decorations.pathreplacing,angles,quotes}
\usetikzlibrary{fadings}
\usepackage{units}
\usepackage{amssymb}
\usepackage{pifont}
\usepackage{nth}
\usepackage{mathtools}
\usepackage[draft]{tikzpeople}
\usepackage{gensymb}
\usepackage{bm}
\usepackage{pgfplots}
\usepackage{pgfmath}
\pgfplotsset{compat=1.8}
\usetikzlibrary{shapes.geometric}

%% Defining Stuff
\newcommand{\PreserveBackslash}[1]{\let\temp=\\#1\let\\=\temp}
\newcolumntype{C}[1]{>{\PreserveBackslash\centering}p{#1}}
\newcolumntype{R}[1]{>{\PreserveBackslash\raggedleft}p{#1}}
\newcolumntype{L}[1]{>{\PreserveBackslash\raggedright}p{#1}}
\newenvironment{cols}[1][]{}{}
\newenvironment{col}[1]{\begin{minipage}{#1}\ignorespaces}{%
\end{minipage}
\ifhmode\unskip\fi
\aftergroup\useignorespacesandallpars}

\def\useignorespacesandallpars#1\ignorespaces\fi{%
#1\fi\ignorespacesandallpars}
\makeatletter
\def\ignorespacesandallpars{%
  \@ifnextchar\par
    {\expandafter\ignorespacesandallpars\@gobble}%
    {}%
}
\makeatother
\tolerance=1
\emergencystretch=\maxdimen
\hyphenpenalty=10000
\hbadness=10000
\setlist{topsep=0pt, leftmargin=*}
\setlist[2]{itemsep=5pt}


\setlength{\parindent}{0pt}
\setlength{\parskip}{1em}

\tcbuselibrary{skins}


\usepackage[dvipsnames]{xcolor}



\usepackage{sectsty}
\usepackage{fontawesome5}

\def\doubleunderline#1{\underline{\underline{#1}}}
\AtBeginDocument{\frontmatter}
\usepackage{bookmark}
\IfFileExists{xurl.sty}{\usepackage{xurl}}{} % add URL line breaks if available
\urlstyle{same}
\hypersetup{
  hidelinks,
  pdfcreator={LaTeX via pandoc}}

\author{}
\date{\vspace{-2.5em}}

\begin{document}

{
\setcounter{tocdepth}{1}
\tableofcontents
}
\mainmatter

\chapter*{Introduction}\label{introduction}
\addcontentsline{toc}{chapter}{Introduction}

Intro

\chapterfont{\color{white}}

\chapter{Trigonometric Equations}\label{trigonometric-equations}

\vspace{-12cm}
\begin{center}
\begin{tikzpicture}
\draw[white] (10,5) circle (0.01);
\draw[DarkOrchid,fill=DarkOrchid] (0,0) rectangle (18.2,0.2);
\draw[DarkOrchid,fill=DarkOrchid] (6,1.8) rectangle (18.2,0.2);
\node[white] at (12.1,1) {\Huge{\textsc{Trigonometric Equations}}};
\node at (3,1) {\Large{\textsc{Chapter 15}}};
\end{tikzpicture}
\end{center}

Each part of an operation of the form \(a^n\) can be described using the following terminology:

\begin{center}
    \begin{tikzpicture}
        \node at (0,0) {\large{$a^n$}};
        \draw[stealth-,DarkOrchid,thick] (0.3,0.2) --++(1,0.5) node[right] {"power" or "exponent"};
        \draw[stealth-,DarkOrchid,thick] (-0.32,0) --++(-1,0.2) node[left] {"base"};
    \end{tikzpicture}
\end{center}

\emph{Power functions} take the form \(f(x)=x^n\), with \(x\) as the \textcolor{DarkOrchid}{\textit{base}} and a constant power, \(n\).

\[\text{e.g.}\quad f(x)=x^3\]

\emph{Exponential functions} take the form \(f(x)=a^x\), with \(x\) as the \textcolor{DarkOrchid}{\textit{exponent}} and a constant base, \(a>0,a\ne 1\).

\[\text{e.g.}\quad f(x)=3^x\]

An example of an exponential function in everyday life is that of something \emph{appreciating} by a percentage of its value, such as an antique vase of value £4000 increasing by 20\% each year. Its value after \(0\) years, \(1\) years, \(2\) years, and so on, can be calculated as follows:

\begin{align*}
    \color{black!25}4000\times 1.2^0\;=\;&4000 && \color{DarkOrchid}\longleftarrow\text{After 0 years}\\[0.5em]
    4000\times 1.2^{\color{DarkOrchid}1}\color{black}\;=\;&4800 && \color{DarkOrchid}\longleftarrow\text{After 1 year}\\[0.5em]
    4000\times 1.2^{\color{DarkOrchid}2}\color{black}\;=\;&5760 && \color{DarkOrchid}\longleftarrow\text{After 2 years}\\[0.5em]
    4000\times 1.2^{\color{DarkOrchid}3}\color{black}\;=\;&6912 && \color{DarkOrchid}\longleftarrow\text{After 3 years}\\[0.5em]
    4000\times 1.2^{\color{DarkOrchid}4}\color{black}\;=\;&8294.40 && \color{DarkOrchid}\longleftarrow\text{After 4 years}
\end{align*}

The function to describe its value \(V\) after \(\color{DarkOrchid}x\) years is given by: \(V(\color{DarkOrchid}x\color{black})=4000\times1.2^{\color{DarkOrchid}x}\)

One advantage of defining this function is the ability to calculate the value at times other that after whole years. For example, the value after \textcolor{DarkOrchid}{three and a half years} can be calculated as:

\[V(\color{DarkOrchid}3.5\color{black})=4000\times1.2^{\color{DarkOrchid}3.5}=7571.72\]

This chapter will introduce a range of skills required when working with exponential functions.

\pagebreak

\section{Graphs of Exponential Functions}\label{graphs-of-exponential-functions}

\vspace{-0.5cm}

\setlength{\columnsep}{30pt}

\begin{multicols}{2}

Where $a>1$:

$y=a^x$ is \textit{strictly increasing} on $x\in\mathbb{R}$:

\vspace{-0.5cm}
\begin{center}
    \begin{tikzpicture}[scale=0.8]
        \draw[-stealth] (-5,0) -- (5,0) node[below] {$x$};
        \draw[-stealth] (0,-1) -- (0,4.5) node[left] {$y$};
        \node[below left] {O};
        \draw[smooth,thick,DarkOrchid,domain=-5:5] plot (\x,{1.3^(\x)}) node[above left] {$y=a^x$};
        \draw[DarkOrchid,fill=DarkOrchid] (0,1) circle (0.05) node[above left] {$1$};
        \draw[DarkOrchid,fill=DarkOrchid] (1,1.3) circle (0.05) node[above] {$(1,a)$};
    \end{tikzpicture}
\end{center}

\vspace{-0.5cm}
This describes \textbf{exponential growth}.

\columnbreak

Where $0<a<1$:

$y=a^x$ is \textit{strictly decreasing} on $x\in\mathbb{R}$:

\vspace{-0.5cm}
\begin{center}
    \begin{tikzpicture}[scale=0.8]
        \draw[-stealth] (-5,0) -- (5,0) node[below] {$x$};
        \draw[-stealth] (0,-1) -- (0,4.5) node[left] {$y$};
        \node[below left] {O};
        \draw[smooth,thick,DarkOrchid,domain=-5:5] plot (\x,{0.76923^(\x)}) node[above] {$y=a^x$};
        \draw[DarkOrchid,fill=DarkOrchid] (0,1) circle (0.05) node[above left] {$1$};
        \draw[DarkOrchid,fill=DarkOrchid] (1,0.76923) circle (0.05) node[above] {$(1,a)$};
    \end{tikzpicture}
\end{center}

\vspace{-0.5cm}
This describes \textbf{exponential decay}.

\end{multicols}

\setlength{\columnsep}{10pt}

Since \(a^0=1\), any graph of the form \(y=a^x\) will pass through the point \((0,1)\) for all \(a\ne 0\).

Since \(a^1=a\), any graph of the form \(y=a^x\) will pass through the point \((1,a)\) for all \(a\).

Determining the equation of the graph of an exponential function can typically be achieved using substitution or consideration of graph transformations, along with knowledge of points \((0,1)\) and \((1,a)\).

\begin{tcolorbox}[title=Example,colback=DarkOrchid!2!, colframe=DarkOrchid]

The graph of $y=a^x+b$ is shown below. 

\vspace{-0.5cm}
\begin{center}
    \begin{tikzpicture}
        \draw[-stealth] (-5,0) -- (5,0) node[below] {$x$};
        \draw[-stealth] (0,-1) -- (0,4) node[left] {$y$};
        \node[below left] {O};
        \draw[smooth,thick,domain=-4:4,yscale=0.2] plot (\x,{2^(\x)+1}) node[above] {$y=a^x+b$};
        \draw[fill] (0,0.4) circle (0.05) node[above left] {$2$};
        \draw[fill] (2,1) circle (0.05) node[above left] {$(2,5)$};
    \end{tikzpicture}
\end{center}

\tcblower

\vspace{-0.5cm}
\begin{align*}
    2&=a^0+b && \color{DarkOrchid}\longleftarrow\text{Substitute }(0,2)\\[0.5em]
    2&=1+b &&\\[0.5em]
    1&=b && \color{DarkOrchid}\longleftarrow\text{Solve to obtain }b\\[1em]
    5&=a^2+1 && \color{DarkOrchid}\longleftarrow\text{Substitute }(2,5)\text{ and }b=1\\[0.5em]
    4&=a^2 &&\\[0.5em]
    2&=a && \color{DarkOrchid}\longleftarrow\text{Solve to obtain }a\\[1em]
    y&=2^x+1&&\color{DarkOrchid}\longleftarrow\text{State equation}
\end{align*}

\end{tcolorbox}

\pagebreak

\subsection*{Exercise 15.1}\label{exercise-15.1}
\addcontentsline{toc}{subsection}{Exercise 15.1}

\begin{enumerate}
    \item Find the equation of each exponential graphs using the form given.
    \begin{enumerate}
        \begin{multicols}{2}
            \item[(a)] $y=a^x$\\
            \begin{tikzpicture}[scale=0.8]
            \draw[-stealth] (-4,0) -- (4,0) node[below] {$x$};
            \draw[-stealth] (0,-2) -- (0,4) node[left] {$y$};
            \draw[smooth,thick,domain=-4:4,yscale=0.2] plot (\x,{2^(\x)+1});
            \draw[fill] (0,0.4) circle (0.05) node[above left] {$2$};
            \draw[fill] (2,1) circle (0.05) node[above left] {$(2,5)$};
            \end{tikzpicture}
            \item[(c)] $y=a^x$\\
            \begin{tikzpicture}[scale=0.8]
            \draw[-stealth] (-4,0) -- (4,0) node[below] {$x$};
            \draw[-stealth] (0,-2) -- (0,4) node[left] {$y$};
            \draw[smooth,thick,domain=-4:4,yscale=0.2] plot (\x,{2^(\x)+1});
            \draw[fill] (0,0.4) circle (0.05) node[above left] {$2$};
            \draw[fill] (2,1) circle (0.05) node[above left] {$(2,5)$};
            \end{tikzpicture}
            \item[(e)] $y=a^x$\\
            \begin{tikzpicture}[scale=0.8]
            \draw[-stealth] (-4,0) -- (4,0) node[below] {$x$};
            \draw[-stealth] (0,-2) -- (0,4) node[left] {$y$};
            \draw[smooth,thick,domain=-4:4,yscale=0.2] plot (\x,{2^(\x)+1});
            \draw[fill] (0,0.4) circle (0.05) node[above left] {$2$};
            \draw[fill] (2,1) circle (0.05) node[above left] {$(2,5)$};
            \end{tikzpicture}
            \item[(g)] $y=a^x$\\
            \begin{tikzpicture}[scale=0.8]
            \draw[-stealth] (-4,0) -- (4,0) node[below] {$x$};
            \draw[-stealth] (0,-2) -- (0,4) node[left] {$y$};
            \draw[smooth,thick,domain=-4:4,yscale=0.2] plot (\x,{2^(\x)+1});
            \draw[fill] (0,0.4) circle (0.05) node[above left] {$2$};
            \draw[fill] (2,1) circle (0.05) node[above left] {$(2,5)$};
            \end{tikzpicture}
            \item[(b)] $y=a^x$\\
            \begin{tikzpicture}[scale=0.8]
            \draw[-stealth] (-4,0) -- (4,0) node[below] {$x$};
            \draw[-stealth] (0,-2) -- (0,4) node[left] {$y$};
            \draw[smooth,thick,domain=-4:4,yscale=0.2] plot (\x,{2^(\x)+1});
            \draw[fill] (0,0.4) circle (0.05) node[above left] {$2$};
            \draw[fill] (2,1) circle (0.05) node[above left] {$(2,5)$};
            \end{tikzpicture}
            \item[(d)] $y=a^x$\\
            \begin{tikzpicture}[scale=0.8]
            \draw[-stealth] (-4,0) -- (4,0) node[below] {$x$};
            \draw[-stealth] (0,-2) -- (0,4) node[left] {$y$};
            \draw[smooth,thick,domain=-4:4,yscale=0.2] plot (\x,{2^(\x)+1});
            \draw[fill] (0,0.4) circle (0.05) node[above left] {$2$};
            \draw[fill] (2,1) circle (0.05) node[above left] {$(2,5)$};
            \end{tikzpicture}
            \item[(f)] $y=a^x$\\
            \begin{tikzpicture}[scale=0.8]
            \draw[-stealth] (-4,0) -- (4,0) node[below] {$x$};
            \draw[-stealth] (0,-2) -- (0,4) node[left] {$y$};
            \draw[smooth,thick,domain=-4:4,yscale=0.2] plot (\x,{2^(\x)+1});
            \draw[fill] (0,0.4) circle (0.05) node[above left] {$2$};
            \draw[fill] (2,1) circle (0.05) node[above left] {$(2,5)$};
            \end{tikzpicture}
            \item[(h)] $y=a^x$\\
            \begin{tikzpicture}[scale=0.8]
            \draw[-stealth] (-4,0) -- (4,0) node[below] {$x$};
            \draw[-stealth] (0,-2) -- (0,4) node[left] {$y$};
            \draw[smooth,thick,domain=-4:4,yscale=0.2] plot (\x,{2^(\x)+1});
            \draw[fill] (0,0.4) circle (0.05) node[above left] {$2$};
            \draw[fill] (2,1) circle (0.05) node[above left] {$(2,5)$};
            \end{tikzpicture}
        \end{multicols}
    \end{enumerate}
\end{enumerate}

\pagebreak

\section{Logarithms}\label{logarithms}

\chapterfont{\color{white}}

\chapter*{Answers}\label{answers}
\addcontentsline{toc}{chapter}{Answers}

\vspace{-10.5cm}

\begin{center}
    \begin{tikzpicture}
        \draw[white] (10,5) circle (0.01);
        \draw[black!50,fill=black!50] (0,0) rectangle (18.2,0.2);
        \draw[black!50,fill=black!50] (6,1.8) rectangle (18.2,0.2);
        \node[white] at (12.1,1) {\Huge{\textsc{Answers}}};
        \node[white] at (4,1) {\Large{\textsc{Chapter 13}}};
    \end{tikzpicture}
\end{center}

\chapterfont{\color{white}}

\chapter*{Challenge Problems}\label{challenge-problems}
\addcontentsline{toc}{chapter}{Challenge Problems}

\vspace{-10.5cm}

\begin{center}
    \begin{tikzpicture}
        \draw[white] (10,5) circle (0.01);
        \draw[black!50,fill=black!50] (0,0) rectangle (18.2,0.2);
        \draw[black!50,fill=black!50] (6,1.8) rectangle (18.2,0.2);
        \node[white] at (12.1,1) {\Huge{\textsc{Challenge Problems}}};
        \node[white] at (4,1) {\Large{\textsc{Chapter 13}}};
    \end{tikzpicture}
\end{center}

The following problems \textbf{do not} represent the kind of question expected to feature in a Higher Mathematics exam, either in the way they are presented or the level of difficulty. Instead, they aim to encourage a flexible approach towards problem-solving and an understanding that the skills covered in the course have applications beyond those featured in any typical exam. \emph{Some questions may be solveable without using the skills covered in this chapter, and some questions may be unreleated to this chapter.}

\bibliography{book.bib,packages.bib}

\end{document}
