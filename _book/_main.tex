% Options for packages loaded elsewhere
\PassOptionsToPackage{unicode}{hyperref}
\PassOptionsToPackage{hyphens}{url}
\documentclass[
  12pt,
  openany]{book}
\usepackage{xcolor}
\usepackage{amsmath,amssymb}
\setcounter{secnumdepth}{5}
\usepackage{iftex}
\ifPDFTeX
  \usepackage[T1]{fontenc}
  \usepackage[utf8]{inputenc}
  \usepackage{textcomp} % provide euro and other symbols
\else % if luatex or xetex
  \usepackage{unicode-math} % this also loads fontspec
  \defaultfontfeatures{Scale=MatchLowercase}
  \defaultfontfeatures[\rmfamily]{Ligatures=TeX,Scale=1}
\fi
\usepackage{lmodern}
\ifPDFTeX\else
  % xetex/luatex font selection
  \setmainfont[]{Times New Roman}
\fi
% Use upquote if available, for straight quotes in verbatim environments
\IfFileExists{upquote.sty}{\usepackage{upquote}}{}
\IfFileExists{microtype.sty}{% use microtype if available
  \usepackage[]{microtype}
  \UseMicrotypeSet[protrusion]{basicmath} % disable protrusion for tt fonts
}{}
\makeatletter
\@ifundefined{KOMAClassName}{% if non-KOMA class
  \IfFileExists{parskip.sty}{%
    \usepackage{parskip}
  }{% else
    \setlength{\parindent}{0pt}
    \setlength{\parskip}{6pt plus 2pt minus 1pt}}
}{% if KOMA class
  \KOMAoptions{parskip=half}}
\makeatother
\usepackage{longtable,booktabs,array}
\usepackage{calc} % for calculating minipage widths
% Correct order of tables after \paragraph or \subparagraph
\usepackage{etoolbox}
\makeatletter
\patchcmd\longtable{\par}{\if@noskipsec\mbox{}\fi\par}{}{}
\makeatother
% Allow footnotes in longtable head/foot
\IfFileExists{footnotehyper.sty}{\usepackage{footnotehyper}}{\usepackage{footnote}}
\makesavenoteenv{longtable}
\setlength{\emergencystretch}{3em} % prevent overfull lines
\providecommand{\tightlist}{%
  \setlength{\itemsep}{0pt}\setlength{\parskip}{0pt}}
\usepackage[]{natbib}
\bibliographystyle{apalike}
\usepackage{booktabs}
\usepackage[top=1cm,bottom=1cm,left=1.5cm,right=1.5cm,includehead,a4paper]{geometry}
\usepackage{fancyhdr}
  \pagestyle{fancy}
  \fancypagestyle{plain}{%
  \renewcommand{\headrulewidth}{0pt}%
  \fancyhf{}%
}
  \fancyhf{}
  \fancyhf[HRO,HLE]{\thepage}
  \fancyhf[HC]{\leftmark}

%% Book Structure
\usepackage{booktabs}
\usepackage{titlesec}
\usepackage{caption}
\usepackage{subcaption}

\usepackage[no-math]{fontspec}
\usepackage{lmodern}
\renewcommand{\familydefault}{lmss}
\usepackage{amsthm}
\usepackage{mathtools}
\usepackage{unicode-math}

%% Utilities
\usepackage{tcolorbox}
\usepackage{hyperref}
\usepackage{tabularx}
\usepackage{multirow}
\usepackage{array}
\usepackage{wrapfig}
\graphicspath{ {images/} }
\usepackage{exsheets}
\usepackage{setspace}
\usepackage{etoolbox}
\usepackage{enumitem}
\usepackage{graphbox,graphicx}
\usepackage{float}
\usepackage{adjustbox}
\usepackage{quotchap}
\usepackage{dirtytalk}
\usepackage{setspace}
\usepackage{multicol}
\usepackage{epigraph}
\usepackage[switch, modulo]{lineno}
\usepackage{vwcol}

%% Maths and Tikz
\usepackage{tikz}
\usepackage{tikzlings}
\usetikzlibrary{math}
\usetikzlibrary{patterns}
\usetikzlibrary{decorations.pathreplacing,angles,quotes}
\usetikzlibrary{fadings}
\usepackage{units}
\usepackage{amssymb}
\usepackage{pifont}
\usepackage{nth}
\usepackage{mathtools}
\usepackage[draft]{tikzpeople}
\usepackage{gensymb}
\usepackage{bm}
\usepackage{pgfplots}
\usepackage{pgfmath}
\pgfplotsset{compat=1.8}
\usetikzlibrary{shapes.geometric}

%% Defining Stuff
\newcommand{\PreserveBackslash}[1]{\let\temp=\\#1\let\\=\temp}
\newcolumntype{C}[1]{>{\PreserveBackslash\centering}p{#1}}
\newcolumntype{R}[1]{>{\PreserveBackslash\raggedleft}p{#1}}
\newcolumntype{L}[1]{>{\PreserveBackslash\raggedright}p{#1}}
\newenvironment{cols}[1][]{}{}
\newenvironment{col}[1]{\begin{minipage}{#1}\ignorespaces}{%
\end{minipage}
\ifhmode\unskip\fi
\aftergroup\useignorespacesandallpars}

\def\useignorespacesandallpars#1\ignorespaces\fi{%
#1\fi\ignorespacesandallpars}
\makeatletter
\def\ignorespacesandallpars{%
  \@ifnextchar\par
    {\expandafter\ignorespacesandallpars\@gobble}%
    {}%
}
\makeatother
\tolerance=1
\emergencystretch=\maxdimen
\hyphenpenalty=10000
\hbadness=10000
\setlist{topsep=0pt, leftmargin=*}
\setlist[2]{itemsep=5pt}


\setlength{\parindent}{0pt}
\setlength{\parskip}{1em}

\tcbuselibrary{skins}


\usepackage[dvipsnames]{xcolor}



\usepackage{sectsty}
\usepackage{fontawesome5}

\def\doubleunderline#1{\underline{\underline{#1}}}
\AtBeginDocument{\frontmatter}
\usepackage{bookmark}
\IfFileExists{xurl.sty}{\usepackage{xurl}}{} % add URL line breaks if available
\urlstyle{same}
\hypersetup{
  hidelinks,
  pdfcreator={LaTeX via pandoc}}

\author{}
\date{\vspace{-2.5em}}

\begin{document}

{
\setcounter{tocdepth}{1}
\tableofcontents
}
\mainmatter

\chapter*{Introduction}\label{introduction}
\addcontentsline{toc}{chapter}{Introduction}

Intro

\chapter{Placeholder}\label{placeholder}

\chapter{Placeholder}\label{placeholder-1}

\chapter{Placeholder}\label{placeholder-2}

\chapter{Placeholder}\label{placeholder-3}

\chapter{Placeholder}\label{placeholder-4}

\chapter{Placeholder}\label{placeholder-5}

\chapter{Placeholder}\label{placeholder-6}

\chapter{Placeholder}\label{placeholder-7}

\chapter{Placeholder}\label{placeholder-8}

\chapter{Placeholder}\label{placeholder-9}

\chapter{Placeholder}\label{placeholder-10}

\chapter{Placeholder}\label{placeholder-11}

\chapter{Placeholder}\label{placeholder-12}

\chapter{Placeholder}\label{placeholder-13}

\chapter{Placeholder}\label{placeholder-14}

\chapter{Placeholder}\label{placeholder-15}

\chapter{Placeholder}\label{placeholder-16}

\chapter{Placeholder}\label{placeholder-17}

\chapter{Placeholder}\label{placeholder-18}

\chapter{Placeholder}\label{placeholder-19}

\chapter{Placeholder}\label{placeholder-20}

\chapter{Placeholder}\label{placeholder-21}

\chapterfont{\color{white}}

\chapter{Trig Equations}\label{trig-equations}

\vspace{-12cm}
\begin{center}
\begin{tikzpicture}
\draw[white] (10,5) circle (0.01);
\draw[DarkOrchid,fill=DarkOrchid] (0,0) rectangle (18.2,0.2);
\draw[DarkOrchid,fill=DarkOrchid] (6,1.8) rectangle (18.2,0.2);
\node[white] at (12.1,1) {\Huge{\textsc{Trig Equations}}};
\node at (3,1) {\Large{\textsc{Chapter 23}}};
\end{tikzpicture}
\end{center}

Different types of equations (such as \emph{quadratic} or \emph{linear}) require different methods in order to solve them.

\emph{Trigonometric equations} are those which include \(\sin{x\degree}\), \(\cos{x\degree}\) or \(\tan{x\degree}\). Since these are some of the more complicated operations encountered so far, the best way to begin thinking about how trig equations may be solved is to consider a \emph{graphical} interpretation:

\vspace{-0.8cm}

\begin{align*}
\color{DarkOrchid}\text{Equation:}&\quad \sin{x\degree}=0.5\\[0.5em]
\color{DarkOrchid}\text{Interpretation:}&\quad \textit{``For which value(s) of }x\textit{ is the graph }y=\sin{x}\textit{ at a height of }0.5\textit{?''}
\end{align*}

\vspace{-0.3cm}

Plotting the graph of \(y=\sin{x\degree}\) and marking the points at which its height is \(0.5\) reveals that, between \(0\degree\) and \(360\degree\), there are \textbf{two solutions}:

\begin{center}
    \begin{tikzpicture}[xscale=0.04,yscale=3]
        \foreach \x in {30,60,...,360}{
        \draw[black!30] (\x,-1) -- (\x,1);
        }
        \foreach \y in {-1,-0.9,...,1}{
        \draw[black!30] (-0.1,\y) -- (360,\y);
        }
        \draw[black!30] (0,1) -- (360,1);
        \draw[-stealth] (0,0) -- (400,0) node[below] {$x\degree$};
        \draw[-stealth] (0,-1.1) -- (0,1.2) node[left] {$y$};
        \foreach \x in {0,30,...,360}{
        \draw (\x,0) -- (\x,-0.01);
        }
        \foreach \x in {90,180,270,360}{
        \node[below] at (\x,0) {\x \degree};
        }
        \node[left] at (0,1) {1};
        \node[left] at (0,0.5) {0.5};
        \node[left] at (0,0) {0};
        \node[left] at (0,-0.5) {-0.5};
        \node[left] at (0,-1) {-1};
        \draw[thick,DarkOrchid,smooth,domain=0:360] plot (\x,{sin(\x)});
        \node[DarkOrchid,above] at (90,1) {$y=\sin{x}\degree$};
        \draw[DarkOrchid,dashed] (30,0.5) -- (30,0) node[below] {$30\degree$};
        \draw[DarkOrchid,dashed] (150,0.5) -- (150,0) node[below] {$150\degree$};
        \draw[fill,DarkOrchid] (30,0.5) ellipse (3 and 0.04);
        \draw[fill,DarkOrchid] (150,0.5) ellipse (3 and 0.04);
    \end{tikzpicture}
\end{center}

Hence the solutions to the equation \(\sin{x\degree}=0.5\) where \(0<x<360\) are \(\color{DarkOrchid}x\degree=30\degree\) and \(\color{DarkOrchid}x\degree=150\degree\).

Note that \(\sin^{-1}(0.5)\) gives the \emph{acute} solution \(30\degree\), and \(180\degree-30\degree=150\degree\) using the graph's symmetry.

The same above graph can similarly be used to solve a number of equations in \(\sin{x\degree}\):

\vspace{-0.8cm}

\begin{align*}
\color{DarkOrchid}\text{Equation:}&\quad \sin{x\degree}=-1\\[0.5em]
\color{DarkOrchid}\text{Interpretation:}&\quad \textit{``For which value(s) of }x\textit{ is the graph }y=\sin{x}\textit{ at a height of }-1\textit{?''}\\[0.5em]
\color{DarkOrchid}\text{Solution:}&\quad x\degree=270\degree
\end{align*}

\vspace{-0.3cm}

The graph also reveals that equations of the form \(\sin{x\degree}=k\) only have solutions for \(-1\leqslant k \leqslant 1\):

\vspace{-0.8cm}

\begin{align*}
\color{DarkOrchid}\text{Equation:}&\quad \sin{x\degree}=2\\[0.5em]
\color{DarkOrchid}\text{Interpretation:}&\quad \textit{``For which value(s) of }x\textit{ is the graph }y=\sin{x}\textit{ at a height of }2\textit{?''}\\[0.5em]
\color{DarkOrchid}\text{Solution:}&\quad \textbf{No solutions.}
\end{align*}

\vspace{-0.3cm}

This chapter will introduce a method for solving trig equations without a graph, and explore contexts in which the ability to solve trig equations may be useful in the real world.

\pagebreak

\section{Solving Trig Equations Graphically}\label{solving-trig-equations-graphically}

\vspace{-0.5cm}

When considering \(\sin{x\degree}=0.4\), calculating \(\sin^{-1}(0.4)\) gives the \textcolor{DarkOrchid}{\textbf{acute angle}} \(23.6\degree\) (to 1 decimal place).

This acute angle and the symmetries of \(y=\sin{x\degree}\) allow \(\sin{x\degree}=0.4\) and \(\sin{x\degree}=-0.4\) to be solved:

\vspace{-0.7cm}

\begin{center}
    \begin{tikzpicture}[xscale=0.04,yscale=1.6]
        \draw[-stealth] (0,0) -- (380,0) node[below] {$x\degree$};
        \draw[-stealth] (0,-1.1) -- (0,1.2) node[above] {$y$};
        \foreach \x in {0,30,...,360}{
        \draw (\x,0) -- (\x,-0.01);
        }
        \foreach \x in {90,180,270,360}{
        \node[below] at (\x,0) {\small{\x \degree}};
        }
        \node[left] at (0,1) {1};
        \node[left] at (0,0.4) {0.4};
        \node[left] at (0,0) {0};
        \node[left] at (0,-0.4) {-0.4};
        \node[left] at (0,-1) {-1};
        \draw[very thick,DarkOrchid,smooth,domain=0:360] plot (\x,{sin(\x)});
        \node[DarkOrchid,above right] at (360,0) {$y=\sin{x}\degree$};
        \draw[DarkOrchid,dashed] (23.6,0.4) -- (23.6,0) node[below] {$23.6\degree$};
        \draw[DarkOrchid,dashed] (156.4,0.4) -- (156.4,0) node[below] {$156.4\degree$};
        \draw[DarkOrchid,dashed] (203.6,-0.4) -- (203.6,0);
        \node[above,DarkOrchid] at (203.6,0) {$203.6\degree$};
        \node[above,DarkOrchid] at (336.4,0) {$336.4\degree$};
        \draw[DarkOrchid,dashed] (336.4,-0.40) -- (336.4,0);
        \draw[fill,DarkOrchid] (23.6,0.4) ellipse (1.75 and 0.032);
        \draw[fill,DarkOrchid] (156.4,0.4) ellipse (1.75 and 0.032);
        \draw[fill,DarkOrchid] (180+23.6,-0.4) ellipse (1.75 and 0.032);
        \draw[fill,DarkOrchid] (360-23.6,-0.4) ellipse (1.75 and 0.032);
        \draw (90,-1) -- (90,1);
        \draw (180,-1) -- (180,1);
        \draw (270,-1) -- (270,1);
        \draw (360,-1) -- (360,1);
        \draw[thick,DarkOrchid,-stealth] (2,0.4) -- (16.4,0.4);
        \draw[thick,DarkOrchid,stealth-] (163.4,0.4) -- (178,0.4);
        \draw[thick,DarkOrchid,-stealth] (182,-0.4) -- (196.6,-0.4);
        \draw[thick,DarkOrchid,stealth-] (343.4,-0.4) -- (358,-0.4);
        \node[above] at (45,1) {1st quadrant};
        \node[above,DarkOrchid] at (45,1.3) {$23.6\degree$};
        \node[above] at (135,1) {2nd quadrant};
        \node[above,DarkOrchid] at (135,1.3) {$180\degree-23.6\degree$};
        \node[above] at (225,1) {3rd quadrant};
        \node[above,DarkOrchid] at (225,1.3) {$180\degree+23.6\degree$};
        \node[above] at (315,1) {4th quadrant};
        \node[above,DarkOrchid] at (315,1.3) {$360\degree-23.6\degree$};
    \end{tikzpicture}
\end{center}

\vspace{-1.3cm}

\begin{multicols}{2}

\begin{align*}
\text{\color{DarkOrchid}Equation:}\hspace{0.2cm} \sin{x\degree}&=0.4\\[0.1em]
\text{\color{DarkOrchid}Solutions:}\hspace{0.8cm} x\degree&=23.6\degree,156.4\degree
\end{align*}

\vfill\null

\columnbreak

\begin{align*}
\text{\color{DarkOrchid}Equation:}\hspace{0.2cm} \sin{x\degree}&=-0.4\\[0.1em]
\text{\color{DarkOrchid}Solutions:}\hspace{0.75cm} x\degree&=203.6\degree,336.4\degree
\end{align*}

\vfill\null

\end{multicols}

\vspace{-1cm}

Similarly, \(\cos{x\degree}=0.7\) and \(\cos{x\degree}=-0.7\) can be solved using \textcolor{DarkOrchid}{\textbf{acute angle}} \(\cos^{-1}(0.7)=45.6\degree\) (1dp):

\vspace{-0.7cm}

\begin{center}
    \begin{tikzpicture}[xscale=0.04,yscale=1.6]
        \draw[-stealth] (0,0) -- (380,0) node[below] {$x\degree$};
        \draw[-stealth] (0,-1.1) -- (0,1.2) node[above] {$y$};
        \foreach \x in {0,30,...,360}{
        \draw (\x,0) -- (\x,-0.01);
        }
        \foreach \x in {90,180,270,360}{
        \node[below] at (\x,0) {\small{\x \degree}};
        }
        \node[left] at (0,1) {1};
        \node[left] at (0,0.7) {0.7};
        \node[left] at (0,0) {0};
        \node[left] at (0,-0.7) {-0.7};
        \node[left] at (0,-1) {-1};
        \draw[very thick,DarkOrchid,smooth,domain=0:360] plot (\x,{cos(\x)});
        \node[DarkOrchid,right] at (360,1) {$y=\cos{x}\degree$};
        \draw[DarkOrchid,dashed] (45.6,0.7) -- (45.6,0) node[below] {$45.6\degree$};
        \draw[DarkOrchid,dashed] (314.4,0.7) -- (314.4,0) node[below] {$314.4\degree$};
        \draw[DarkOrchid,dashed] (134.4,-0.7) -- (134.4,0);
        \node[above,DarkOrchid] at (134.6,0) {$134.6\degree$};
        \node[above,DarkOrchid] at (222.6,0) {$222.6\degree$};
        \draw[DarkOrchid,dashed] (222.6,-0.7) -- (222.6,0);
        \draw[fill,DarkOrchid] (45.6,0.7) ellipse (1.75 and 0.032);
        \draw[fill,DarkOrchid] (314.4,0.7) ellipse (1.75 and 0.032);
        \draw[fill,DarkOrchid] (180+45.6,-0.7) ellipse (1.75 and 0.032);
        \draw[fill,DarkOrchid] (180-45.6,-0.7) ellipse (1.75 and 0.032);
        \draw (90,-1) -- (90,1);
        \draw (180,-1) -- (180,1);
        \draw (270,-1) -- (270,1);
        \draw (360,-1) -- (360,1);
        \draw[thick,DarkOrchid,-stealth] (2,0.7) -- (37.6,0.7);
        \draw[thick,DarkOrchid,stealth-] (142.4,-0.7) -- (178,-0.7);
        \draw[thick,DarkOrchid,-stealth] (182,-0.7) -- (217.6,-0.7);
        \draw[thick,DarkOrchid,stealth-] (322.4,0.7) -- (358,0.7);
        \node[above] at (45,1) {1st quadrant};
        \node[above,DarkOrchid] at (45,1.3) {$45.6\degree$};
        \node[above] at (135,1) {2nd quadrant};
        \node[above,DarkOrchid] at (135,1.3) {$180\degree-45.6\degree$};
        \node[above] at (225,1) {3rd quadrant};
        \node[above,DarkOrchid] at (225,1.3) {$180\degree+45.6\degree$};
        \node[above] at (315,1) {4th quadrant};
        \node[above,DarkOrchid] at (315,1.3) {$360\degree-45.6\degree$};
    \end{tikzpicture}
\end{center}

\vspace{-1.3cm}

\begin{multicols}{2}

\begin{align*}
\text{\color{DarkOrchid}Equation:}\hspace{0.2cm} \cos{x\degree}&=0.7\\[0.1em]
\text{\color{DarkOrchid}Solutions:}\hspace{0.8cm} x\degree&=45.6\degree,314.4\degree
\end{align*}

\vfill\null

\columnbreak

\begin{align*}
\text{\color{DarkOrchid}Equation:}\hspace{0.2cm} \cos{x\degree}&=-0.7\\[0.1em]
\text{\color{DarkOrchid}Solutions:}\hspace{0.75cm} x\degree&=134.4\degree,225.6\degree
\end{align*}

\vfill\null

\end{multicols}

\vspace{-1cm}

To solve \(\tan{x\degree}=1.3\) and \(\tan{x\degree}=-1.3\), again the \textcolor{DarkOrchid}{\textbf{acute angle}} \(\tan^{-1}(1.3)=52.4\degree\) (1dp) is helpful:

\vspace{-0.7cm}

\begin{center}
    \begin{tikzpicture}[xscale=0.04,yscale=0.4]
        \draw[-stealth] (0,0) -- (380,0) node[below] {$x\degree$};
        \draw[-stealth] (0,-4) -- (0,4.8) node[above] {$y$};
        \foreach \x in {0,30,...,360}{
        \draw (\x,0) -- (\x,-0.01);
        }
        \foreach \x in {90,180,270,360}{
        \node[below] at (\x,0) {\small{\x \degree}};
        }
        \node[left] at (0,4) {4};
        \node[left] at (0,1.3) {1.3};
        \node[left] at (0,0) {0};
        \node[left] at (0,-1.3) {-1.3};
        \node[left] at (0,-4) {-4};
        \draw[very thick,DarkOrchid,smooth,domain=0:75] plot (\x,{tan(\x)});
        \draw[very thick,DarkOrchid,smooth,domain=105:255] plot (\x,{tan(\x)});
        \draw[very thick,DarkOrchid,smooth,domain=285:360] plot (\x,{tan(\x)});
        \node[DarkOrchid,right] at (360,1) {$y=\tan{x}\degree$};
        \draw[DarkOrchid,dashed] (52.4,1.3) -- (52.4,0) node[below] {$52.4\degree$};
        \draw[DarkOrchid,dashed] (307.6,-1.3) -- (307.6,0) node[above] {$307.6\degree$};
        \draw[DarkOrchid,dashed] (127.6,-1.3) -- (127.6,0);
        \node[above,DarkOrchid] at (127.6,0) {$127.6\degree$};
        \node[below,DarkOrchid] at (232.4,0) {$232.6\degree$};
        \draw[DarkOrchid,dashed] (232.4,-1.3) -- (232.4,0);
        \draw[fill,DarkOrchid] (52.4,1.3) ellipse (1.75 and 4*0.032);
        \draw[fill,DarkOrchid] (307.6,-1.3) ellipse (1.75 and 4*0.032);
        \draw[fill,DarkOrchid] (180+52.4,1.3) ellipse (1.75 and 4*0.032);
        \draw[fill,DarkOrchid] (180-52.4,-1.3) ellipse (1.75 and 4*0.032);
        \draw (90,-4) -- (90,4);
        \draw (180,-4) -- (180,4);
        \draw (270,-4) -- (270,4);
        \draw (360,-4) -- (360,4);
        \draw[thick,DarkOrchid,-stealth] (2,1.3) -- (42.6,1.3);
        \draw[thick,DarkOrchid,stealth-] (137.4,-1.3) -- (178,-1.3);
        \draw[thick,DarkOrchid,-stealth] (182,1.3) -- (222.6,1.3);
        \draw[thick,DarkOrchid,stealth-] (317.4,-1.3) -- (358,-1.3);
        \node[above] at (45,4) {1st quadrant};
        \node[above,DarkOrchid] at (45,4*1.3) {$52.4\degree$};
        \node[above] at (135,4) {2nd quadrant};
        \node[above,DarkOrchid] at (135,4*1.3) {$180\degree-52.4\degree$};
        \node[above] at (225,4) {3rd quadrant};
        \node[above,DarkOrchid] at (225,4*1.3) {$180\degree+52.4\degree$};
        \node[above] at (315,4) {4th quadrant};
        \node[above,DarkOrchid] at (315,4*1.3) {$360\degree-52.4\degree$};
    \end{tikzpicture}
\end{center}

\vspace{-1.3cm}

\begin{multicols}{2}

\begin{align*}
\text{\color{DarkOrchid}Equation:}\hspace{0.2cm} \tan{x\degree}&=1.3\\[0.1em]
\text{\color{DarkOrchid}Solutions:}\hspace{0.8cm} x\degree&=52.4\degree,232.4\degree
\end{align*}

\vfill\null

\columnbreak

\begin{align*}
\text{\color{DarkOrchid}Equation:}\hspace{0.2cm} \tan{x\degree}&=-1.3\\[0.1em]
\text{\color{DarkOrchid}Solutions:}\hspace{0.75cm} x\degree&=127.6\degree,307.6\degree
\end{align*}

\vfill\null

\end{multicols}

\vspace{-1cm}

\pagebreak

\subsection*{Exercise 23.1}\label{exercise-23.1}
\addcontentsline{toc}{subsection}{Exercise 23.1}

\begin{enumerate}
    \item Use the graph of $y=\sin{x\degree}$ below to solve the following equations for $0\leqslant x \leqslant 360$.
    \begin{center}
    \begin{tikzpicture}[xscale=0.025,yscale=2]
        \draw[-stealth] (0,0) -- (380,0) node[right] {$x$};
        \draw[-stealth] (0,-1) -- (0,1.2) node[above] {$y$};
        \foreach \x in {0,30,...,360}{
        \draw (\x,0) -- (\x,-0.01);
        }
        \foreach \x in {90,180,270,360}{
        \node[below] at (\x,0) {$\small{\x}$};
        }
        \node[left] at (0,1) {1};
        \node[left] at (0,0.6) {0.6};
        \node[left] at (0,0) {0};
        \node[left] at (0,-1) {-1};
        \node[below] at (36.9,0) {$36.9$};
        \draw[thick,smooth,domain=0:360] plot (\x,{sin(\x)});
        \node[above] at (90,1) {$y=\sin{x}\degree$};
        \draw[dashed] (0,0.6) -- (36.9,0.6) -- (36.9,0);
        \draw[fill] (36.9,0.6) ellipse (1.5 and 0.01875);
        \node[below right] at (450,1.4) {
        \begin{minipage}{8cm}
        \begin{enumerate}
                \item[(a)] $\sin{x\degree}=1$
                \item[(b)] $\sin{x\degree}=0$
                \item[(c)] $\sin{x\degree}=-1$
                \item[(d)] $\sin{x\degree}=0.6$
                \item[(e)] $\sin{x\degree}=-0.6$
                \item[(f)] $\sin{x\degree}=1.1$
        \end{enumerate}
        \end{minipage}
        };
    \end{tikzpicture}
\end{center}
    \item Use the graph of $y=\cos{x\degree}$ below to solve the following equations for $0\leqslant x \leqslant 360$.
    \begin{center}
    \begin{tikzpicture}[xscale=0.025,yscale=2]
        \draw[-stealth] (0,0) -- (380,0) node[right] {$x$};
        \draw[-stealth] (0,-1) -- (0,1.2) node[above] {$y$};
        \foreach \x in {0,30,...,360}{
        \draw (\x,0) -- (\x,-0.01);
        }
        \foreach \x in {90,180,270,360}{
        \node[below] at (\x,0) {$\small{\x}$};
        }
        \node[left] at (0,1.05) {1};
        \node[left] at (0,0.85) {0.9};
        \node[left] at (0,0) {0};
        \node[left] at (0,-1) {-1};
        \node[below] at (25.8,0) {$25.8$};
        \draw[thick,smooth,domain=0:360] plot (\x,{cos(\x)});
        \node[above] at (90,0.7) {$y=\cos{x}\degree$};
        \draw[dashed] (0,0.9) -- (25.8,0.9) -- (25.8,0);
        \draw[fill] (25.8,0.9) ellipse (1.5 and 0.01875);
        \node[below right] at (450,1.4) {
        \begin{minipage}{8cm}
        \begin{enumerate}
                \item[(a)] $\cos{x\degree}=1$
                \item[(b)] $\cos{x\degree}=0$
                \item[(c)] $\cos{x\degree}=-1$
                \item[(d)] $\cos{x\degree}=0.9$
                \item[(e)] $\cos{x\degree}=-0.9$
                \item[(f)] $\cos{x\degree}=1.1$
        \end{enumerate}
        \end{minipage}
        };
    \end{tikzpicture}
\end{center}
    \item Use the graph of $y=\tan{x\degree}$ below to solve the following equations for $0\leqslant x \leqslant 360$.
    \begin{center}
    \begin{tikzpicture}[xscale=0.025,yscale=0.4]
        \draw[-stealth] (0,0) -- (380,0) node[right] {$x$};
        \draw[-stealth] (0,-6) -- (0,6.5) node[above] {$y$};
        \foreach \x in {0,30,...,360}{
        \draw (\x,0) -- (\x,-0.01);
        }
        \foreach \x in {90,180,270,360}{
        \node[below] at (\x,0) {$\small{\x}$};
        }
        \node[left] at (0,6) {6};
        \node[left] at (0,1) {1};
        \node[left] at (0,0) {0};
        \node[left] at (0,-6) {-6};
        \node[below] at (45,0) {$45$};
        \draw[thick,smooth,domain=0:80] plot (\x,{tan(\x)});
        \draw[thick,smooth,domain=100:260] plot (\x,{tan(\x)});
        \draw[thick,smooth,domain=280:360] plot (\x,{tan(\x)});
        \draw[dashed] (90,-6) -- (90,6);
        \draw[dashed] (270,-6) -- (270,6);
        \node[left] at (260,6) {$y=\tan{x}\degree$};
        \draw[dashed] (0,1) -- (45,1) -- (45,0);
        \draw[fill] (45,1) ellipse (1.5 and 4*0.01875);
        \node[below right] at (450,6.6) {
        \begin{minipage}{8cm}
        \begin{enumerate}
                \item[(a)] $\tan{x\degree}=1$
                \item[(b)] $\tan{x\degree}=0$
                \item[(c)] $\tan{x\degree}=-1$
                \item[(d)] $\tan{x\degree}=3$
                \item[(e)] $\tan{x\degree}=-3$
                \item[(f)] $\tan{x\degree}=17$
        \end{enumerate}
        \end{minipage}
        };
    \end{tikzpicture}
\end{center}
    \item Given that $\sin^{-1}\left(\frac{1}{3}\right)=19.5\degree$, to one decimal place, solve $\sin{x\degree}=\frac{1}{3}$ for $0\leqslant x \leqslant 360$.
    \item Given that $\cos^{-1}\left(\frac{3}{4}\right)=41.4\degree$, to one decimal place, solve $4\cos{x\degree}=3$ for $0\leqslant x \leqslant 360$.
    \item Given that $\sin^{-1}\left(\frac{2}{3}\right)=41.8\degree$, to one decimal place, solve $3\sin{x\degree}+2=0$ for $0\leqslant x \leqslant 360$.
    \item Solve $5\cos{x\degree}+3=0$ for $0\leqslant x \leqslant 720$.
\end{enumerate}

\pagebreak

\section{Using the ASTC Diagram}\label{using-the-astc-diagram}

\newpage

\subsection*{Exercise 23.2}\label{exercise-23.2}
\addcontentsline{toc}{subsection}{Exercise 23.2}

\newpage

\section{Points of Intersection}\label{points-of-intersection}

\newpage

\subsection*{Exercise 23.3}\label{exercise-23.3}
\addcontentsline{toc}{subsection}{Exercise 23.3}

\newpage

\section{Trig Functions in Context}\label{trig-functions-in-context}

\newpage

\subsection*{Exercise 23.4}\label{exercise-23.4}
\addcontentsline{toc}{subsection}{Exercise 23.4}

\newpage

\section{Trig Identities}\label{trig-identities}

\newpage

\subsection*{Exercise 23.5}\label{exercise-23.5}
\addcontentsline{toc}{subsection}{Exercise 23.5}

\newpage

\section*{Review Exercise}\label{review-exercise}
\addcontentsline{toc}{section}{Review Exercise}

\chapterfont{\color{white}}

\chapter*{Answers}\label{answers}
\addcontentsline{toc}{chapter}{Answers}

\vspace{-10.5cm}

\begin{center}
    \begin{tikzpicture}
        \draw[white] (10,5) circle (0.01);
        \draw[black!50,fill=black!50] (0,0) rectangle (18.2,0.2);
        \draw[black!50,fill=black!50] (6,1.8) rectangle (18.2,0.2);
        \node[white] at (12.1,1) {\Huge{\textsc{Answers}}};
        \node[white] at (4,1) {\Large{\textsc{Chapter 13}}};
    \end{tikzpicture}
\end{center}

\chapterfont{\color{white}}

\chapter*{Challenge Problems}\label{challenge-problems}
\addcontentsline{toc}{chapter}{Challenge Problems}

\vspace{-10.5cm}

\begin{center}
    \begin{tikzpicture}
        \draw[white] (10,5) circle (0.01);
        \draw[black!50,fill=black!50] (0,0) rectangle (18.2,0.2);
        \draw[black!50,fill=black!50] (6,1.8) rectangle (18.2,0.2);
        \node[white] at (12.1,1) {\Huge{\textsc{Challenge Problems}}};
        \node[white] at (4,1) {\Large{\textsc{Chapter 13}}};
    \end{tikzpicture}
\end{center}

The following problems \textbf{do not} represent the kind of question expected to feature in a Higher Mathematics exam, either in the way they are presented or the level of difficulty. Instead, they aim to encourage a flexible approach towards problem-solving and an understanding that the skills covered in the course have applications beyond those featured in any typical exam. \emph{Some questions may be solveable without using the skills covered in this chapter, and some questions may be unreleated to this chapter.}

\bibliography{book.bib,packages.bib}

\end{document}
