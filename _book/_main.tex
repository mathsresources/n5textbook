% Options for packages loaded elsewhere
\PassOptionsToPackage{unicode}{hyperref}
\PassOptionsToPackage{hyphens}{url}
\documentclass[
  12pt,
  openany]{book}
\usepackage{xcolor}
\usepackage{amsmath,amssymb}
\setcounter{secnumdepth}{5}
\usepackage{iftex}
\ifPDFTeX
  \usepackage[T1]{fontenc}
  \usepackage[utf8]{inputenc}
  \usepackage{textcomp} % provide euro and other symbols
\else % if luatex or xetex
  \usepackage{unicode-math} % this also loads fontspec
  \defaultfontfeatures{Scale=MatchLowercase}
  \defaultfontfeatures[\rmfamily]{Ligatures=TeX,Scale=1}
\fi
\usepackage{lmodern}
\ifPDFTeX\else
  % xetex/luatex font selection
  \setmainfont[]{Times New Roman}
\fi
% Use upquote if available, for straight quotes in verbatim environments
\IfFileExists{upquote.sty}{\usepackage{upquote}}{}
\IfFileExists{microtype.sty}{% use microtype if available
  \usepackage[]{microtype}
  \UseMicrotypeSet[protrusion]{basicmath} % disable protrusion for tt fonts
}{}
\makeatletter
\@ifundefined{KOMAClassName}{% if non-KOMA class
  \IfFileExists{parskip.sty}{%
    \usepackage{parskip}
  }{% else
    \setlength{\parindent}{0pt}
    \setlength{\parskip}{6pt plus 2pt minus 1pt}}
}{% if KOMA class
  \KOMAoptions{parskip=half}}
\makeatother
\usepackage{longtable,booktabs,array}
\usepackage{calc} % for calculating minipage widths
% Correct order of tables after \paragraph or \subparagraph
\usepackage{etoolbox}
\makeatletter
\patchcmd\longtable{\par}{\if@noskipsec\mbox{}\fi\par}{}{}
\makeatother
% Allow footnotes in longtable head/foot
\IfFileExists{footnotehyper.sty}{\usepackage{footnotehyper}}{\usepackage{footnote}}
\makesavenoteenv{longtable}
\setlength{\emergencystretch}{3em} % prevent overfull lines
\providecommand{\tightlist}{%
  \setlength{\itemsep}{0pt}\setlength{\parskip}{0pt}}
\usepackage[]{natbib}
\bibliographystyle{apalike}
\usepackage{booktabs}
\usepackage[top=1cm,bottom=1cm,left=1.5cm,right=1.5cm,includehead,a4paper]{geometry}
\usepackage{fancyhdr}
  \pagestyle{fancy}
  \fancypagestyle{plain}{%
  \renewcommand{\headrulewidth}{0pt}%
  \fancyhf{}%
}
  \fancyhf{}
  \fancyhf[HRO,HLE]{\thepage}
  \fancyhf[HC]{\leftmark}

%% Book Structure
\usepackage{booktabs}
\usepackage{titlesec}
\usepackage{caption}
\usepackage{subcaption}

\usepackage[no-math]{fontspec}
\usepackage{lmodern}
\renewcommand{\familydefault}{lmss}
\usepackage{amsthm}
\usepackage{mathtools}
\usepackage{unicode-math}

%% Utilities
\usepackage{tcolorbox}
\usepackage{hyperref}
\usepackage{tabularx}
\usepackage{multirow}
\usepackage{array}
\usepackage{wrapfig}
\graphicspath{ {images/} }
\usepackage{exsheets}
\usepackage{setspace}
\usepackage{etoolbox}
\usepackage{enumitem}
\usepackage{graphbox,graphicx}
\usepackage{float}
\usepackage{adjustbox}
\usepackage{quotchap}
\usepackage{dirtytalk}
\usepackage{setspace}
\usepackage{multicol}
\usepackage{epigraph}
\usepackage[switch, modulo]{lineno}
\usepackage{vwcol}

%% Maths and Tikz
\usepackage{tikz}
\usepackage{tikzlings}
\usetikzlibrary{math}
\usetikzlibrary{patterns}
\usetikzlibrary{decorations.pathreplacing,angles,quotes}
\usetikzlibrary{fadings}
\usepackage{units}
\usepackage{amssymb}
\usepackage{pifont}
\usepackage{nth}
\usepackage{mathtools}
\usepackage[draft]{tikzpeople}
\usepackage{gensymb}
\usepackage{bm}
\usepackage{pgfplots}
\usepackage{pgfmath}
\pgfplotsset{compat=1.8}
\usetikzlibrary{shapes.geometric}

%% Defining Stuff
\newcommand{\PreserveBackslash}[1]{\let\temp=\\#1\let\\=\temp}
\newcolumntype{C}[1]{>{\PreserveBackslash\centering}p{#1}}
\newcolumntype{R}[1]{>{\PreserveBackslash\raggedleft}p{#1}}
\newcolumntype{L}[1]{>{\PreserveBackslash\raggedright}p{#1}}
\newenvironment{cols}[1][]{}{}
\newenvironment{col}[1]{\begin{minipage}{#1}\ignorespaces}{%
\end{minipage}
\ifhmode\unskip\fi
\aftergroup\useignorespacesandallpars}

\def\useignorespacesandallpars#1\ignorespaces\fi{%
#1\fi\ignorespacesandallpars}
\makeatletter
\def\ignorespacesandallpars{%
  \@ifnextchar\par
    {\expandafter\ignorespacesandallpars\@gobble}%
    {}%
}
\makeatother
\tolerance=1
\emergencystretch=\maxdimen
\hyphenpenalty=10000
\hbadness=10000
\setlist{topsep=0pt, leftmargin=*}
\setlist[2]{itemsep=5pt}


\setlength{\parindent}{0pt}
\setlength{\parskip}{1em}

\tcbuselibrary{skins}


\usepackage[dvipsnames]{xcolor}



\usepackage{sectsty}
\usepackage{fontawesome5}

\def\doubleunderline#1{\underline{\underline{#1}}}
\AtBeginDocument{\frontmatter}
\usepackage{bookmark}
\IfFileExists{xurl.sty}{\usepackage{xurl}}{} % add URL line breaks if available
\urlstyle{same}
\hypersetup{
  hidelinks,
  pdfcreator={LaTeX via pandoc}}

\author{}
\date{\vspace{-2.5em}}

\begin{document}

{
\setcounter{tocdepth}{1}
\tableofcontents
}
\mainmatter

\chapter*{Introduction}\label{introduction}
\addcontentsline{toc}{chapter}{Introduction}

Intro

\chapter{Placeholder}\label{placeholder}

\chapter{Placeholder}\label{placeholder-1}

\chapter{Placeholder}\label{placeholder-2}

\chapter{Placeholder}\label{placeholder-3}

\chapter{Placeholder}\label{placeholder-4}

\chapter{Placeholder}\label{placeholder-5}

\chapter{Placeholder}\label{placeholder-6}

\chapter{Placeholder}\label{placeholder-7}

\chapter{Placeholder}\label{placeholder-8}

\chapter{Placeholder}\label{placeholder-9}

\chapter{Placeholder}\label{placeholder-10}

\chapter{Placeholder}\label{placeholder-11}

\chapter{Placeholder}\label{placeholder-12}

\chapter{Placeholder}\label{placeholder-13}

\chapter{Placeholder}\label{placeholder-14}

\chapter{Placeholder}\label{placeholder-15}

\chapter{Placeholder}\label{placeholder-16}

\chapter{Placeholder}\label{placeholder-17}

\chapter{Placeholder}\label{placeholder-18}

\chapter{Placeholder}\label{placeholder-19}

\chapter{Placeholder}\label{placeholder-20}

\chapter{Placeholder}\label{placeholder-21}

\chapterfont{\color{white}}

\chapter{Trig Equations}\label{trig-equations}

\vspace{-12cm}
\begin{center}
\begin{tikzpicture}
\draw[white] (10,5) circle (0.01);
\draw[DarkOrchid,fill=DarkOrchid] (0,0) rectangle (18.2,0.2);
\draw[DarkOrchid,fill=DarkOrchid] (6,1.8) rectangle (18.2,0.2);
\node[white] at (12.1,1) {\Huge{\textsc{Trig Equations}}};
\node at (3,1) {\Large{\textsc{Chapter 23}}};
\end{tikzpicture}
\end{center}

\newpage

\section{Solving Trig Equations Graphically}\label{solving-trig-equations-graphically}

\newpage

\subsection*{Exercise 23.1}\label{exercise-23.1}
\addcontentsline{toc}{subsection}{Exercise 23.1}

\newpage

\section{Using the ASTC Diagram}\label{using-the-astc-diagram}

\newpage

\subsection*{Exercise 23.2}\label{exercise-23.2}
\addcontentsline{toc}{subsection}{Exercise 23.2}

\newpage

\section{Points of Intersection}\label{points-of-intersection}

\newpage

\subsection*{Exercise 23.3}\label{exercise-23.3}
\addcontentsline{toc}{subsection}{Exercise 23.3}

\newpage

\section{Trig Functions in Context}\label{trig-functions-in-context}

\newpage

\subsection*{Exercise 23.4}\label{exercise-23.4}
\addcontentsline{toc}{subsection}{Exercise 23.4}

\newpage

\section{Trig Identities}\label{trig-identities}

\newpage

\subsection*{Exercise 23.5}\label{exercise-23.5}
\addcontentsline{toc}{subsection}{Exercise 23.5}

\newpage

\section*{Review Exercise}\label{review-exercise}
\addcontentsline{toc}{section}{Review Exercise}

\chapterfont{\color{white}}

\chapter*{Answers}\label{answers}
\addcontentsline{toc}{chapter}{Answers}

\vspace{-10.5cm}

\begin{center}
    \begin{tikzpicture}
        \draw[white] (10,5) circle (0.01);
        \draw[black!50,fill=black!50] (0,0) rectangle (18.2,0.2);
        \draw[black!50,fill=black!50] (6,1.8) rectangle (18.2,0.2);
        \node[white] at (12.1,1) {\Huge{\textsc{Answers}}};
        \node[white] at (4,1) {\Large{\textsc{Chapter 13}}};
    \end{tikzpicture}
\end{center}

\chapterfont{\color{white}}

\chapter*{Challenge Problems}\label{challenge-problems}
\addcontentsline{toc}{chapter}{Challenge Problems}

\vspace{-10.5cm}

\begin{center}
    \begin{tikzpicture}
        \draw[white] (10,5) circle (0.01);
        \draw[black!50,fill=black!50] (0,0) rectangle (18.2,0.2);
        \draw[black!50,fill=black!50] (6,1.8) rectangle (18.2,0.2);
        \node[white] at (12.1,1) {\Huge{\textsc{Challenge Problems}}};
        \node[white] at (4,1) {\Large{\textsc{Chapter 13}}};
    \end{tikzpicture}
\end{center}

The following problems \textbf{do not} represent the kind of question expected to feature in a Higher Mathematics exam, either in the way they are presented or the level of difficulty. Instead, they aim to encourage a flexible approach towards problem-solving and an understanding that the skills covered in the course have applications beyond those featured in any typical exam. \emph{Some questions may be solveable without using the skills covered in this chapter, and some questions may be unreleated to this chapter.}

\bibliography{book.bib,packages.bib}

\end{document}
