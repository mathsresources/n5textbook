% Options for packages loaded elsewhere
\PassOptionsToPackage{unicode}{hyperref}
\PassOptionsToPackage{hyphens}{url}
\documentclass[
  12pt,
  openany]{book}
\usepackage{xcolor}
\usepackage{amsmath,amssymb}
\setcounter{secnumdepth}{5}
\usepackage{iftex}
\ifPDFTeX
  \usepackage[T1]{fontenc}
  \usepackage[utf8]{inputenc}
  \usepackage{textcomp} % provide euro and other symbols
\else % if luatex or xetex
  \usepackage{unicode-math} % this also loads fontspec
  \defaultfontfeatures{Scale=MatchLowercase}
  \defaultfontfeatures[\rmfamily]{Ligatures=TeX,Scale=1}
\fi
\usepackage{lmodern}
\ifPDFTeX\else
  % xetex/luatex font selection
  \setmainfont[]{Times New Roman}
\fi
% Use upquote if available, for straight quotes in verbatim environments
\IfFileExists{upquote.sty}{\usepackage{upquote}}{}
\IfFileExists{microtype.sty}{% use microtype if available
  \usepackage[]{microtype}
  \UseMicrotypeSet[protrusion]{basicmath} % disable protrusion for tt fonts
}{}
\makeatletter
\@ifundefined{KOMAClassName}{% if non-KOMA class
  \IfFileExists{parskip.sty}{%
    \usepackage{parskip}
  }{% else
    \setlength{\parindent}{0pt}
    \setlength{\parskip}{6pt plus 2pt minus 1pt}}
}{% if KOMA class
  \KOMAoptions{parskip=half}}
\makeatother
\usepackage{longtable,booktabs,array}
\usepackage{calc} % for calculating minipage widths
% Correct order of tables after \paragraph or \subparagraph
\usepackage{etoolbox}
\makeatletter
\patchcmd\longtable{\par}{\if@noskipsec\mbox{}\fi\par}{}{}
\makeatother
% Allow footnotes in longtable head/foot
\IfFileExists{footnotehyper.sty}{\usepackage{footnotehyper}}{\usepackage{footnote}}
\makesavenoteenv{longtable}
\setlength{\emergencystretch}{3em} % prevent overfull lines
\providecommand{\tightlist}{%
  \setlength{\itemsep}{0pt}\setlength{\parskip}{0pt}}
\usepackage[]{natbib}
\bibliographystyle{apalike}
\usepackage{booktabs}
\usepackage[top=1cm,bottom=1cm,left=1.5cm,right=1.5cm,includehead,a4paper]{geometry}
\usepackage{fancyhdr}
  \pagestyle{fancy}
  \fancypagestyle{plain}{%
  \renewcommand{\headrulewidth}{0pt}%
  \fancyhf{}%
}
  \fancyhf{}
  \fancyhf[HRO,HLE]{\thepage}
  \fancyhf[HC]{\leftmark}

%% Book Structure
\usepackage{booktabs}
\usepackage{titlesec}
\usepackage{caption}
\usepackage{subcaption}

\usepackage[no-math]{fontspec}
\usepackage{lmodern}
\renewcommand{\familydefault}{lmss}
\usepackage{amsthm}
\usepackage{mathtools}
\usepackage{unicode-math}

%% Utilities
\usepackage{tcolorbox}
\usepackage{hyperref}
\usepackage{tabularx}
\usepackage{multirow}
\usepackage{array}
\usepackage{wrapfig}
\graphicspath{ {images/} }
\usepackage{exsheets}
\usepackage{setspace}
\usepackage{etoolbox}
\usepackage{enumitem}
\usepackage{graphbox,graphicx}
\usepackage{float}
\usepackage{adjustbox}
\usepackage{quotchap}
\usepackage{dirtytalk}
\usepackage{setspace}
\usepackage{multicol}
\usepackage{epigraph}
\usepackage[switch, modulo]{lineno}
\usepackage{vwcol}

%% Maths and Tikz
\usepackage{tikz}
\usepackage{tikzlings}
\usetikzlibrary{math}
\usetikzlibrary{patterns}
\usetikzlibrary{decorations.pathreplacing,angles,quotes}
\usetikzlibrary{fadings}
\usepackage{units}
\usepackage{amssymb}
\usepackage{pifont}
\usepackage{nth}
\usepackage{mathtools}
\usepackage[draft]{tikzpeople}
\usepackage{gensymb}
\usepackage{bm}
\usepackage{pgfplots}
\usepackage{pgfmath}
\pgfplotsset{compat=1.8}
\usetikzlibrary{shapes.geometric}

%% Defining Stuff
\newcommand{\PreserveBackslash}[1]{\let\temp=\\#1\let\\=\temp}
\newcolumntype{C}[1]{>{\PreserveBackslash\centering}p{#1}}
\newcolumntype{R}[1]{>{\PreserveBackslash\raggedleft}p{#1}}
\newcolumntype{L}[1]{>{\PreserveBackslash\raggedright}p{#1}}
\newenvironment{cols}[1][]{}{}
\newenvironment{col}[1]{\begin{minipage}{#1}\ignorespaces}{%
\end{minipage}
\ifhmode\unskip\fi
\aftergroup\useignorespacesandallpars}

\def\useignorespacesandallpars#1\ignorespaces\fi{%
#1\fi\ignorespacesandallpars}
\makeatletter
\def\ignorespacesandallpars{%
  \@ifnextchar\par
    {\expandafter\ignorespacesandallpars\@gobble}%
    {}%
}
\makeatother
\tolerance=1
\emergencystretch=\maxdimen
\hyphenpenalty=10000
\hbadness=10000
\setlist{topsep=0pt, leftmargin=*}
\setlist[2]{itemsep=5pt}


\setlength{\parindent}{0pt}
\setlength{\parskip}{1em}

\tcbuselibrary{skins}


\usepackage[dvipsnames]{xcolor}



\usepackage{sectsty}
\usepackage{fontawesome5}

\def\doubleunderline#1{\underline{\underline{#1}}}
\AtBeginDocument{\frontmatter}
\usepackage{bookmark}
\IfFileExists{xurl.sty}{\usepackage{xurl}}{} % add URL line breaks if available
\urlstyle{same}
\hypersetup{
  hidelinks,
  pdfcreator={LaTeX via pandoc}}

\author{}
\date{\vspace{-2.5em}}

\begin{document}

{
\setcounter{tocdepth}{1}
\tableofcontents
}
\mainmatter

\chapter*{Introduction}\label{introduction}
\addcontentsline{toc}{chapter}{Introduction}

Intro

\chapterfont{\color{white}}

\chapter{The Straight Line}\label{the-straight-line}

\vspace{-12cm}
\begin{center}
\begin{tikzpicture}
\draw[white] (10,5) circle (0.01);
\draw[DarkOrchid,fill=DarkOrchid] (0,0) rectangle (18.2,0.2);
\draw[DarkOrchid,fill=DarkOrchid] (8,1.8) rectangle (18.2,0.2);
\node[white] at (13.1,1) {\Huge{\textsc{The Straight Line}}};
\node at (4,1) {\Large{\textsc{Chapter 1}}};
\end{tikzpicture}
\end{center}

\chapterfont{\color{white}}

\chapter{Recurrence Relations}\label{recurrence-relations}

\vspace{-12cm}
\begin{center}
\begin{tikzpicture}
\draw[white] (10,5) circle (0.01);
\draw[DarkOrchid,fill=DarkOrchid] (0,0) rectangle (18.2,0.2);
\draw[DarkOrchid,fill=DarkOrchid] (8,1.8) rectangle (18.2,0.2);
\node[white] at (13.1,1) {\Huge{\textsc{Recurrence Relations}}};
\node at (4,1) {\Large{\textsc{Chapter 2}}};
\end{tikzpicture}
\end{center}

\chapterfont{\color{white}}

\chapter{Differentiation I}\label{differentiation-i}

\vspace{-12cm}
\begin{center}
\begin{tikzpicture}
\draw[white] (10,5) circle (0.01);
\draw[DarkOrchid,fill=DarkOrchid] (0,0) rectangle (18.2,0.2);
\draw[DarkOrchid,fill=DarkOrchid] (8,1.8) rectangle (18.2,0.2);
\node[white] at (13.1,1) {\Huge{\textsc{Differentiation I}}};
\node at (4,1) {\Large{\textsc{Chapter 3}}};
\end{tikzpicture}
\end{center}

\chapterfont{\color{white}}

\chapter{Quadratic Theory}\label{quadratic-theory}

\vspace{-12cm}
\begin{center}
\begin{tikzpicture}
\draw[white] (10,5) circle (0.01);
\draw[DarkOrchid,fill=DarkOrchid] (0,0) rectangle (18.2,0.2);
\draw[DarkOrchid,fill=DarkOrchid] (8,1.8) rectangle (18.2,0.2);
\node[white] at (13.1,1) {\Huge{\textsc{Quadratic Theory}}};
\node at (4,1) {\Large{\textsc{Chapter 4}}};
\end{tikzpicture}
\end{center}

\chapterfont{\color{white}}

\chapter{Sets and Functions}\label{sets-and-functions}

\vspace{-12cm}
\begin{center}
\begin{tikzpicture}
\draw[white] (10,5) circle (0.01);
\draw[DarkOrchid,fill=DarkOrchid] (0,0) rectangle (18.2,0.2);
\draw[DarkOrchid,fill=DarkOrchid] (8,1.8) rectangle (18.2,0.2);
\node[white] at (13.1,1) {\Huge{\textsc{Sets and Functions}}};
\node at (4,1) {\Large{\textsc{Chapter 5}}};
\end{tikzpicture}
\end{center}

\chapterfont{\color{white}}

\chapter{Trigonometry}\label{trigonometry}

\vspace{-12cm}
\begin{center}
\begin{tikzpicture}
\draw[white] (10,5) circle (0.01);
\draw[DarkOrchid,fill=DarkOrchid] (0,0) rectangle (18.2,0.2);
\draw[DarkOrchid,fill=DarkOrchid] (8,1.8) rectangle (18.2,0.2);
\node[white] at (13.1,1) {\Huge{\textsc{Trigonometry}}};
\node at (4,1) {\Large{\textsc{Chapter 6}}};
\end{tikzpicture}
\end{center}

\chapterfont{\color{white}}

\chapter{Graph Transformations}\label{graph-transformations}

\vspace{-10cm}
\begin{center}
\begin{tikzpicture}
\draw[white] (10,5) circle (0.01);
\draw[DarkOrchid,fill=DarkOrchid] (0,0) rectangle (18.2,0.2);
\draw[DarkOrchid,fill=DarkOrchid] (6,1.8) rectangle (18.2,0.2);
\node[white] at (12.1,1) {\Huge{\textsc{Graph Transformations}}};
\node at (3,1) {\Large{\textsc{Chapter 7}}};
\end{tikzpicture}
\end{center}

\chapterfont{\color{white}}

\chapter{Vectors}\label{vectors}

\vspace{-12cm}
\begin{center}
\begin{tikzpicture}
\draw[white] (10,5) circle (0.01);
\draw[DarkOrchid,fill=DarkOrchid] (0,0) rectangle (18.2,0.2);
\draw[DarkOrchid,fill=DarkOrchid] (8,1.8) rectangle (18.2,0.2);
\node[white] at (13.1,1) {\Huge{\textsc{Vectors}}};
\node at (4,1) {\Large{\textsc{Chapter 8}}};
\end{tikzpicture}
\end{center}

\chapterfont{\color{white}}

\chapter{Differentiation II}\label{differentiation-ii}

\vspace{-12cm}
\begin{center}
\begin{tikzpicture}
\draw[white] (10,5) circle (0.01);
\draw[DarkOrchid,fill=DarkOrchid] (0,0) rectangle (18.2,0.2);
\draw[DarkOrchid,fill=DarkOrchid] (8,1.8) rectangle (18.2,0.2);
\node[white] at (13.1,1) {\Huge{\textsc{Differentiation II}}};
\node at (4,1) {\Large{\textsc{Chapter 10}}};
\end{tikzpicture}
\end{center}

\chapterfont{\color{white}}

\chapter{Polynomials}\label{polynomials}

\vspace{-12cm}
\begin{center}
\begin{tikzpicture}
\draw[white] (10,5) circle (0.01);
\draw[DarkOrchid,fill=DarkOrchid] (0,0) rectangle (18.2,0.2);
\draw[DarkOrchid,fill=DarkOrchid] (8,1.8) rectangle (18.2,0.2);
\node[white] at (13.1,1) {\Huge{\textsc{Polynomials}}};
\node at (4,1) {\Large{\textsc{Chapter 10}}};
\end{tikzpicture}
\end{center}

\chapterfont{\color{white}}

\chapter{Integration}\label{integration}

\vspace{-12cm}
\begin{center}
\begin{tikzpicture}
\draw[white] (10,5) circle (0.01);
\draw[blue!70!black!70!,fill=blue!70!black!70!] (0,0) rectangle (18.2,0.2);
\draw[blue!70!black!70!,fill=blue!70!black!70!] (8,1.8) rectangle (18.2,0.2);
\node[white] at (13.1,1) {\Huge{\textsc{Integration}}};
\node at (4,1) {\Large{\textsc{Chapter 11}}};
\end{tikzpicture}
\end{center}

\subsection*{Introduction and Overview}\label{introduction-and-overview}
\addcontentsline{toc}{subsection}{Introduction and Overview}

In mathematics, \textit{integration} can be thought of as the process of calculating some kind of \emph{``sum''}, such as the total area under a curve, by breaking it down into sums of increasingly small parts. This technique was known about and used thousands of years ago in both Ancient Greece and other parts of the world.

\begin{center}
    \begin{multicols}{3}

    \begin{tikzpicture}[scale=0.9]
        \draw[white] (-3,-1) rectangle (3.2,2);
        \draw[fill=blue!20] (-1.6,0) rectangle (-1.2,0.72);
        \draw[fill=blue!20] (-1.2,0) rectangle (-0.8,1.28);
        \draw[fill=blue!20] (-0.8,0) rectangle (-0.4,1.68);
        \draw[fill=blue!20] (-0.4,0) rectangle (0,1.92);
        \draw[fill=blue!20] (0,0) rectangle (0.4,2);
        \draw[fill=blue!20] (0.4,0) rectangle (0.8,1.92);
        \draw[fill=blue!20] (0.8,0) rectangle (1.2,1.68);
        \draw[fill=blue!20] (1.2,0) rectangle (1.6,1.28);
        \draw[fill=blue!20] (1.6,0) rectangle (2,0.72);
        \draw[-stealth,thick] (-3,0) -- (3,0) node[below] {$x$};
        \draw[smooth,domain=-2:2,thick] plot (\x,-0.5*\x*\x+2);
        \node[below] at (0,0) {$\Delta x=0.4$};
    \end{tikzpicture}
    
    \begin{tikzpicture}[scale=0.9]
        \draw[white] (-3,-1) rectangle (3.2,2);
        \foreach \x in {-2,-1.8,...,1.8}{
        \draw[fill=blue!20] (\x,0) rectangle (\x+0.2,-0.5*\x*\x+2);
        }
        \draw[-stealth,thick] (-3,0) -- (3,0) node[below] {$x$};
        \draw[smooth,domain=-2:2,thick] plot (\x,-0.5*\x*\x+2);
        \node[below] at (0,0) {$\Delta x=0.2$};
    \end{tikzpicture}
    
    \begin{tikzpicture}[scale=0.9]
        \draw[white] (-3,-1) rectangle (3.2,2);
        \draw[smooth,domain=-2:2,,thick,fill=blue!20] plot (\x,-0.5*\x*\x+2);
        \draw[-stealth,thick] (-3,0) -- (3,0) node[below] {$x$};
        \node[below] at (0,0) {As $\Delta x\to 0$};
    \end{tikzpicture}
    
    \end{multicols}
\end{center}

In the 17th century both Gottfried Wilhelm Leibniz and Isaac Newton independently discovered the \emph{fundamental theorem of calculus}, showing that \emph{integration} can be performed using \emph{antidifferentiation}, which is the `reverse' of differentiation. Now, \emph{integration} is used to refer to both the idea of calculating such a `sum', and to the process of antidifferentiation.

\begin{center}
    \begin{tikzpicture}
        \draw[very thick,blue!70!black!70!,-stealth] (120:6) arc (120:60:6);
        \draw[very thick,blue!70!black!70!,-stealth] (120:6) arc (120:60:6);
    \end{tikzpicture}
\end{center}

This chapter will first introduce the use of integration to calculate an \emph{antiderivative}, before moving on to using this concept to calculating area enclosed using curves.

\subsection*{Chapter Contents}\label{chapter-contents}
\addcontentsline{toc}{subsection}{Chapter Contents}

\begin{itemize}
    \setlength\itemsep{2em}
    \item[\textcolor{blue!70!black!70!}{\textbullet}] 11.1 Indefinite Integrals
    \item[\textcolor{blue!70!black!70!}{\textbullet}] 11.2 Differential Equations
    \item[\textcolor{blue!70!black!70!}{\textbullet}] 11.3 Definite Integrals
    \item[\textcolor{blue!70!black!70!}{\textbullet}] 11.4 The Area Under a Curve
    \item[\textcolor{blue!70!black!70!}{\textbullet}] 11.5 The Area Between Two Curves
\end{itemize}

Integration skills covered in this chapter will be extended upon in the later chapter of \textbf{Further Calculus}.

\newpage

\section{Indefinite Integrals}\label{indefinite-integrals}

The \textit{derivative} of the function \(f(x)=x^3-4x^2-7x+3\) is obtained by \textit{multiplying by the power} and then \textit{reducing the power by one} for each term.

\vspace{-0.5cm}

\begin{align*}
    \text{If }f(x)&=x^3-4x^2+3\\
    \text{Then }f'(x)&=3x^2-8x
\end{align*}

\vspace{-0.5cm}

\textit{Integration} can be considered the \textit{inverse} of differentiation, with the aim of taking the \textit{derivative} \(f'(x)\) and determining the \textit{original function} \(f(x)\). However, given only the derivative \(f'(x)=3x^2-8x\) it would not be possible to know any constant term the original function \(f(x)\) contained. Finding the \textit{indefinite integral} of a function requires the inclusion of a \textcolor{blue!70!black!70!}{\textbf{Constant of Integration}, $C$}:

\vspace{-0.5cm}

\begin{flalign*}
    \text{If }f'(x)&=3x^2-8x\\
    \text{Then }f(x)&=x^3-4x^2\color{blue!70!black!70!}+C
\end{flalign*}

\vspace{-0.5cm}

\begin{center}
\begin{tcolorbox}[colback=red!5,width=14cm,colframe=red!70!black]
\textbf{Indefinite Integrals:}
\textcolor{black}{\textit{"The (indefinite) integral of }$x^n$\textit{ with respect to }$x$\textit{..."}}
    \begin{center}
        \[\int (x^n)\phantom{.}dx = \frac{x^{n+1}}{n+1}+C\]
    \end{center}
    In other words, \textbf{increase the power by 1} then \textbf{divide by the new power.}
\end{tcolorbox}
\end{center}

The \textit{integral sign}, \(\displaystyle\int\), should always appear accompanied by \(dx\), for a function in \(x\).

Note: \(\displaystyle{\int (f(x)+g(x))}\;dx=\displaystyle{\int f(x)\;dx+\int g(x\;dx)}\)

\begin{tcolorbox}[title=Example 11.1.1,colback=blue!1, colframe=blue!70!black!70!]

Find $\displaystyle{\int (6x^2+8x)}\,dx $.

\tcblower

\vspace{-0.3cm}

\begin{align*}
    &\int (6x^2+8x)\,dx&&\\[0.5em]
    =\;&\frac{6x^3}{3}+\frac{8x^2}{2}+C&&\color{blue!70!black!70!}\longleftarrow \text{Constant of integration}\\[0.5em]
    =\;&2x^3+4x^2+C&&\color{blue!70!black!70!}\longleftarrow \text{Simplify}
\end{align*}

\end{tcolorbox}

\pagebreak

As with differentiation, \textit{preparation for integration} may be needed.

\begin{tcolorbox}[title=Example 11.1.2,colback=blue!1, colframe=blue!70!black!70!]

Find $\displaystyle{\int \left(4\sqrt{x}-\frac{3}{x^2}\right)}\,dx,\;,x>0$.

\tcblower

\vspace{-0.3cm}

\begin{align*}
    &\int \left(4\sqrt{x}-\frac{3}{x^2}\right)\,dx&&\\[0.5em]
    =\;&\int \left(4x^{\frac{1}{2}}-3x^{-2}\right)\,dx&&\color{blue!70!black!70!}\longleftarrow \text{Preparing to integrate}\\[0.5em]
    =\;&\frac{4x^{\frac{3}{2}}}{\frac{3}{2}}-\frac{3x^{-1}}{-1}+C&&\color{blue!70!black!70!}\longleftarrow \text{Integrate: }\frac{1}{2}+1=\frac{1}{2}+\frac{2}{2}=\frac{3}{2}\\[0.5em]
    =\;&\frac{8}{3}x^{\frac{3}{2}}+3x^{-1}+C&&\color{blue!70!black!70!}\longleftarrow \text{Simplify: }4\div\frac{3}{2}=\frac{4}{1}\times\frac{2}{3}=\frac{8}{3}
\end{align*}

\end{tcolorbox}

The \textit{derivative} of a \textit{linear} term is a \textit{constant}, so the integral of a \textit{constant} is \textit{linear}.

\begin{tcolorbox}[title=Example 11.1.3,colback=blue!1, colframe=blue!70!black!70!]

Find $\displaystyle{\int \left((2x-1)(x+3)\right)}\,dx$.

\tcblower

\vspace{-0.3cm}

\begin{align*}
    &\int \left((2x-1)(x+3)\right)\,dx&&\\[0.5em]
    =\;&\int \left(2x^2-5x-3\right)\,dx&&\color{blue!70!black!70!}\longleftarrow \text{Preparing to integrate}\\[0.5em]
    =\;&\frac{2x^3}{3}-\frac{5x^2}{2}-3x+C&&\color{blue!70!black!70!}\longleftarrow \text{Integral of }-3\text{ is }-3x
\end{align*}

\end{tcolorbox}

Integration \textit{with respect to variables other than} \(x\) should not use \(dx\), and the notation used should match the variable of the function instead.

\begin{tcolorbox}[title=Example 11.1.4,colback=blue!1, colframe=blue!70!black!70!]

Find $\displaystyle{\int \left(3t^2-5\right)}\,dt$.

\tcblower

\vspace{-0.3cm}

\begin{align*}
    &\int \left(3t^2-5\right)\,dt&&\\[0.5em]
    =\;&\frac{3t^3}{3}-5t+C&&\\[0.5em]
    =\;&t^3-5t+C&&
\end{align*}

\end{tcolorbox}

\pagebreak

\section{Differential Equations}\label{differential-equations}

\pagebreak

\section{Definite Integrals}\label{definite-integrals}

\vspace{-0.5cm}

The indefinite integral of the function \(f(x)\) can be notated as \(F(x)\):

\[\int f(x)\;dx=F(x)\]

The \textit{definite integral} of \(f(x)\) from \(a\) to \(b\) is the difference between \(F(b)\) and \(F(a)\):

\[\int_a^b f(x)\;dx=F(b)-F(a)\]

Together, they form a core part of the \textbf{Fundamental Theorem of Calculus}:

\begin{center}
\begin{tcolorbox}[colback=red!5,width=14cm,colframe=red!70!black]
\textbf{Fundamental Theorem of Calculus:}

\[\displaystyle{\int_a^b f(x)\;dx=F(b)-F(a)}\]
\centering
where 
\[\displaystyle{F(x) = \int f(x)\;dx}\]

\end{tcolorbox}
\end{center}

Since any \textit{constant of integration} within \(F(x)\) will cancel through subtraction (\(C-C\)), it is \textit{not included} when calculating a \textit{definite integral}.

\begin{tcolorbox}[title=Example 11.3.1,colback=blue!1, colframe=blue!70!black!70!]

Find $\displaystyle{\int_1^2 \left(3x^2+6x-2\right)}\,dx$.

\tcblower

\vspace{-0.3cm}

\begin{align*}
    &\displaystyle{\int_1^2 \left(3x^2+6x-2\right)}\,dx&&\\[0.5em]
    =\;&\left[\frac{3x^3}{3}+\frac{6x^2}{2}-2x\right]_1^2&&\color{blue!70!black!70!}\longleftarrow \text{Integrate}\\[0.5em]
    =\;&\left[x^3+3x^2-2x\right]_1^2&&\color{blue!70!black!70!}\longleftarrow \text{Simplify}\\[0.5em]
    =\;&\left((2)^3+3(2)^2-2(2)\right)-\left((1)^3+5(1)^2-2(1)\right)&&\color{blue!70!black!70!}\longleftarrow \text{Substitute}\\[0.5em]
    =\;&16-4\\[0.5em]
    =\;&12&&\color{blue!70!black!70!}\longleftarrow \text{Evaluate}
\end{align*}

\end{tcolorbox}

\pagebreak

\section{The Area Under a Curve}\label{the-area-under-a-curve}

\vspace{-0.5cm}

The area shaded below, with each \textit{"bar"} having width \(1\) unit, is given by the \textit{sum} \(1+2+3+1+2\).

\begin{center}
    \begin{tikzpicture}
        \draw[fill=blue!20,domain=1:3,smooth] (0.6,0) rectangle (1.2,0.6);
        \draw[fill=blue!20,domain=1:3,smooth] (1.2,0) rectangle (1.8,1.2);
        \draw[fill=blue!20,domain=1:3,smooth] (1.8,0) rectangle (2.4,1.8);
        \draw[fill=blue!20,domain=1:3,smooth] (2.4,0) rectangle (3,0.6);
        \draw[fill=blue!20,domain=1:3,smooth] (3,0) rectangle (3.6,1.2);
        \draw[-stealth] (-1,0) -- (4.5,0) node[below] {$x$};
        \draw[=stealth] (0,-0.5) -- (0,2.5) node[left] {$y$};
        \foreach \x in {1,2,...,6}{
        \node[below] at (\x*0.6,0) {\x};
        }
        \node[left] at (0,0.6) {$1$};
        \node[left] at (0,1.2) {$2$};
        \node[left] at (0,1.8) {$3$};
        \node[below left] {$0$};
    \end{tikzpicture}
\end{center}

An integral, \(\displaystyle{\int}\), can be seen as \textit{"sum"}, but for a continuous function. Given a function \(f(x)\), the area enclosed between a section of the curve \(y=f(x)\) and the \(x\)-axis can be calculated using a definite integral.

\begin{center}
\begin{tcolorbox}[colback=red!5,width=16cm,colframe=red!70!black]
\textbf{Area Under a Curve:}

\vspace{0.2cm}

\begin{center}
    \begin{tikzpicture}
        \fill[blue!20,domain=1:3,smooth] (3,0) -- (1,0) -- (1,1) -- (3,1.6) -- cycle;
        \fill[red!5,domain=1:3,smooth] plot (\x,0.1*\x*\x-0.1*\x+1);
        \draw[-stealth] (-1,0) -- (5,0) node[below] {$x$};
        \draw[=stealth] (0,-0.5) -- (0,3) node[left] {$y$};
        \draw[smooth,domain=-0.3:
        4,yscale=0.1] plot (\x,\x*\x-\x+10) node[right] {$y=f(x)$};
        \draw[dashed] (1,0) -- (1,1);
        \draw[dashed] (3,0) -- (3,1.6);
        \node at (1,-0.3) {$a$};
        \node at (3,-0.3) {$b$};
        \node at (10,1.2) {Area $\displaystyle{=\int_a^b f(x)\;dx=F(b)-F(a)}$};
    \end{tikzpicture}
\end{center}

\end{tcolorbox}
\end{center}

The values of \(a\) and \(b\) can be referred to as the \textit{bounds} of the integration.

\begin{tcolorbox}[title=Example 11.4.1,colback=blue!1, colframe=blue!70!black!70!]

Part of the graph of $y=5+2x-x^2$ is shown. Calculate the shaded area.

\begin{center}
    \begin{tikzpicture}
        \fill[blue!20,smooth,domain=-1:2] plot (\x,0.25*5+0.5*\x-0.25*\x*\x);
        \fill[blue!20] (-1,0) -- (-1,0.5) -- (2,1.25) -- (2,0) -- cycle; 
        \draw[-stealth] (-2.5,0) -- (4.5,0) node[below] {$x$};
        \draw[-stealth] (0,-1) -- (0,2) node[left] {$y$};
        \node[below left] {$0$};
        \draw[smooth,domain=-1.8:3.8,yscale=0.25] plot (\x,5+2*\x-\x*\x) node[below] {$y=5+2x-x^2$};
        \draw (-1,0) -- (-1,0.5);
        \draw (2,0) -- (2,1.25); 
        \node at (-1,-0.3) {$-1$};
        \node at (2,-0.3) {$2$};
    \end{tikzpicture}
\end{center}

\tcblower


\begin{align*}
    &\displaystyle{\int_{-1}^2 \left(5+2x-x^2\right)}\,dx&&\color{blue!70!black!70!}\longleftarrow \text{Area}\\[0.5em]
    =\;&\left[5x+x^2-\frac{x^3}{3}\right]_{-1}^2&&\color{blue!70!black!70!}\longleftarrow \text{Integrate}\\[0.5em]
    =\;&\left(5(2)+(2)^2-\frac{(2)^3}{3}\right)-\left(5(-1)+(-1)^2-\frac{(-1)^3}{3}\right)&&\color{blue!70!black!70!}\longleftarrow \text{Substitute}\\[0.5em]
    =\;&\frac{34}{3}-\left(-\frac{11}{3}\right)&&\color{blue!70!black!70!}\longleftarrow \text{Evaluate}\\[0.5em]
    =\;&15\text{ square units}
\end{align*}

\end{tcolorbox}

\newpage

Where the area between a curve and the \(x\)-axis lies \textit{under} the \(x\)-axis, the definite integral will be negative.

\begin{center}
    \begin{tikzpicture}
        \fill[green!25,smooth,domain=1:3,yscale=0.25] plot (\x,\x*\x*\x-10*\x*\x+27*\x-18);
        \fill[red!25,smooth,domain=3:6,yscale=0.25] plot (\x,\x*\x*\x-10*\x*\x+27*\x-18);
        \draw[-stealth] (-0.5,0) -- (7.5,0) node[below] {$x$};
        \draw[-stealth] (0,-2) -- (0,2.5) node[left] {$y$};
        \draw[smooth,domain=0.6:6.2,yscale=0.25] plot (\x,\x*\x*\x-10*\x*\x+27*\x-18) node[right] {$y=f(x)$};
        \draw[stealth-] (2.8,0.6) -- (3.5,1.6) node[above] {$\displaystyle{\int_a^b f(x)\;dx=A_1}$};
        \node at (2,0.5) {$A_1$};
        \node at (4.5,-1) {$A_2$};
        \draw[stealth-] (6,-1) -- (7,-1.5) node[right] {$\displaystyle{\int_b^c f(x)\;dx=-A_2}$};
        \draw[fill] (1,0) circle (0.03);
        \draw[fill] (3,0) circle (0.03);
        \draw[fill] (6,0) circle (0.03);
        \node at (1.1,-0.2) {\small{$a$}};
        \node at (2.9,-0.2) {\small{$b$}};
        \node at (6.1,-0.2) {\small{$c$}};
    \end{tikzpicture}
\end{center}

To avoid \textit{"positve"} and \textit{"negative"} areas cancelling each other out, such sections must be calculated as separate integrals, and their \textit{absolute values} added.

\[\text{Area }=A_1+A_2\]

\begin{tcolorbox}[title=Example 11.4.2,colback=blue!1, colframe=blue!70!black!70!]

Part of the graph of $y=x^3-10x^2+27x-18$ is shown. Calculate the shaded area.

\begin{center}
    \begin{tikzpicture}
        \fill[blue!20,smooth,domain=1:3,yscale=0.25] plot (\x,\x*\x*\x-10*\x*\x+27*\x-18);
        \fill[blue!20,smooth,domain=3:6,yscale=0.25] plot (\x,\x*\x*\x-10*\x*\x+27*\x-18);
        \draw[-stealth] (-0.5,0) -- (7.5,0) node[below] {$x$};
        \draw[-stealth] (0,-2) -- (0,2) node[left] {$y$};
        \draw[smooth,domain=0.6:6.2,yscale=0.25] plot (\x,\x*\x*\x-10*\x*\x+27*\x-18) node[above right] {$y=x^3-10x^2+27x-18$};
        \draw[fill] (1,0) circle (0.03);
        \draw[fill] (3,0) circle (0.03);
        \draw[fill] (6,0) circle (0.03);
        \node at (1.1,-0.2) {\small{$1$}};
        \node at (2.9,-0.2) {\small{$3$}};
        \node at (6.1,-0.2) {\small{$6$}};
    \end{tikzpicture}
\end{center}

\tcblower


\begin{align*}
    &\displaystyle{\int_1^3 \left(x^3-10x^2+27x-18\right)}\,dx&&\displaystyle{\int_3^6 \left(x^3-10x^2+27x-18\right)}\,dx\\[0.5em]
    &=\left[\frac{x^4}{4}-\frac{10x^3}{3}+\frac{27x^2}{2}-18x\right]_1^3&&=\left[\frac{x^4}{4}-\frac{10x^3}{3}+\frac{27x^2}{2}-18x\right]_3^6\\[0.5em]
    &=\left(\frac{(3)^4}{4}-\frac{10(3)^3}{3}+\frac{27(3)^2}{2}-18(3)\right)&&=\left(\frac{(6)^4}{4}-\frac{10(6)^3}{3}+\frac{27(6)^2}{2}-18(6)\right)\\[0.5em]
    &-\left(\frac{(1)^4}{4}-\frac{10(1)^3}{3}+\frac{27(1)^2}{2}-18(1)\right)&&-\left(\frac{(3)^4}{4}-\frac{10(3)^3}{3}+\frac{27(3)^2}{2}-18(3)\right)\\[0.5em]
    &=-\frac{9}{4}-\left(-\frac{91}{12}\right)&&=-18-\left(-\frac{9}{4}\right)\\[0.5em]
    &=\frac{16}{3}&&=-\frac{63}{4}
\end{align*}
\vspace{0.3cm}
\hspace{3cm}
$\therefore\text{ Area }=\dfrac{16}{3}+\dfrac{63}{4}=\dfrac{253}{12}\text{ square units}$

\end{tcolorbox}

\pagebreak

\section{The Area Between Two Curves}\label{the-area-between-two-curves}

\vspace{-0.5cm}

The area \textbf{between} two curves can be calculated using the subtraction of one definite integral from another:

\begin{multicols}{3}

\begin{center}
    \begin{tikzpicture}
        \fill[blue!20,domain=1:3,smooth] (3,0) -- (1,0) -- (1,1) -- (3,1.6) -- cycle;
        \fill[blue!20,domain=1:3,smooth] plot (\x,-0.1*\x*\x+0.7*\x+0.4);
        \draw[-stealth] (-0.5,0) -- (4.5,0) node[below] {$x$};
        \draw[=stealth] (0,-0.5) -- (0,2.5) node[left] {$y$};
        \draw[smooth,domain=-0.3:
        3.5,yscale=0.1] plot (\x,\x*\x-\x+10) node[above right] {$g(x)$};
        \draw[smooth,domain=-0.3:
        3.5,yscale=0.1] plot (\x,-\x*\x+7*\x+4) node[right] {$f(x)$};
        \draw[dashed] (1,0) -- (1,1);
        \draw[dashed] (3,0) -- (3,1.6);
        \node at (1,-0.3) {$a$};
        \node at (3,-0.3) {$b$};
        \node at (2,-1.3) {$\displaystyle{\int_a^b f(x)\;dx}$};
    \end{tikzpicture}
\end{center}

    \begin{center}
    \begin{tikzpicture}
        \fill[blue!20,domain=1:3,smooth] (3,0) -- (1,0) -- (1,1) -- (3,1.6) -- cycle;
        \fill[pattern color = blue!70!black!70!,pattern=north west lines,domain=1:3,smooth] (3,0) -- (1,0) -- (1,1) -- (3,1.6) -- cycle;
        \fill[white,domain=1:3,smooth] plot (\x,0.1*\x*\x-0.1*\x+1);
        \draw[-stealth] (-0.5,0) -- (4.5,0) node[below] {$x$};
        \draw[=stealth] (0,-0.5) -- (0,2.5) node[left] {$y$};
        \draw[smooth,domain=-0.3:
        3.5,yscale=0.1] plot (\x,\x*\x-\x+10) node[above right] {$g(x)$};
        \draw[smooth,domain=-0.3:
        3.5,yscale=0.1] plot (\x,-\x*\x+7*\x+4) node[right] {$f(x)$};
        \draw[dashed] (1,0) -- (1,1);
        \draw[dashed] (3,0) -- (3,1.6);
        \node at (1,-0.3) {$a$};
        \node at (3,-0.3) {$b$};
        \node at (2,-1.3) {$\displaystyle{\int_a^b g(x)\;dx}$};
    \end{tikzpicture}
\end{center}

\begin{center}
    \begin{tikzpicture}
        \fill[blue!20,domain=1:3,smooth] plot (\x,-0.1*\x*\x+0.7*\x+0.4);
        \fill[blue!20,domain=1:3,smooth] plot (\x,0.1*\x*\x-0.1*\x+1);
        \draw[-stealth] (-0.5,0) -- (4.5,0) node[below] {$x$};
        \draw[=stealth] (0,-0.5) -- (0,2.5) node[left] {$y$};
        \draw[smooth,domain=-0.3:
        3.5,yscale=0.1] plot (\x,\x*\x-\x+10) node[above right] {$g(x)$};
        \draw[smooth,domain=-0.3:
        3.5,yscale=0.1] plot (\x,-\x*\x+7*\x+4) node[right] {$f(x)$};
        \draw[dashed] (1,0) -- (1,1);
        \draw[dashed] (3,0) -- (3,1.6);
        \node at (1,-0.3) {$a$};
        \node at (3,-0.3) {$b$};
        \node at (2,-1.3) {$\displaystyle{\int_a^b f(x)\;dx-\int_a^b g(x)\;dx}$};
    \end{tikzpicture}
\end{center}

\end{multicols}

\vspace{-0.2cm}

Note that \(\displaystyle{\int_a^b f(x)\;dx-\int_a^b g(x)\;dx=\int_a^b (f(x)-g(x))\;dx}\), leading to the following formula:

\begin{center}
\begin{tcolorbox}[colback=red!5,width=16cm,colframe=red!70!black]

\textbf{Area Between Two Curves:}

\vspace{-0.5cm}

\begin{center}
    \begin{tikzpicture}
        \fill[blue!20,domain=1:3,smooth] plot (\x,-0.1*\x*\x+0.7*\x+0.4);
        \fill[blue!20,domain=1:3,smooth] plot (\x,0.1*\x*\x-0.1*\x+1);
        \draw[-stealth] (-0.5,0) -- (4.5,0) node[below] {$x$};
        \draw[=stealth] (0,-0.5) -- (0,2.5) node[left] {$y$};
        \draw[smooth,domain=-0.3:
        3.5,yscale=0.1] plot (\x,\x*\x-\x+10) node[above right] {$g(x)$ $\longleftarrow$ Lower function};
        \draw[smooth,domain=-0.3:
        3.5,yscale=0.1] plot (\x,-\x*\x+7*\x+4) node[right] {$f(x)$ $\longleftarrow$ Upper function};
        \draw[dashed] (1,0) -- (1,1);
        \draw[dashed] (3,0) -- (3,1.6);
        \node at (1,-0.3) {$a$};
        \node at (3,-0.3) {$b$};
        \node at (2,-1.4) {Area $=\displaystyle{\int_a^b (f(x)-g(x))\;dx}$ \quad\textbf{or}\quad Area $=\displaystyle{\int_a^b (\text{Upper}-\text{Lower})\;dx}$};
        \node at (-6,1) {\phantom{.}};
    \end{tikzpicture}
\end{center}

\end{tcolorbox}
\end{center}

\begin{tcolorbox}[title=Example 11.5.1,colback=blue!1, colframe=blue!70!black!70!]

Part of the graphs of $y=-3x$ and $y=4-x^2$ are shown. Calculate the shaded area.

\begin{center}
    \begin{tikzpicture}
        \fill[blue!20,domain=-1:4] plot (\x,0.4-0.1*\x*\x);
        \draw[-stealth] (-3,0) -- (4.5,0) node[below] {$x$};
        \draw[-stealth] (0,-1.4) -- (0,1.4) node[left] {$y$};
        \draw[smooth,domain=-2.5:4.2,yscale=0.1] plot (\x,4-\x*\x);
        \draw[smooth,domain=-2.1:4.2,yscale=0.1] plot (\x,-3*\x);
        \node[above left] at (-2.1,0.63) {$y=-3x$};
        \node[below left] at (-2.5,-0.225) {$y=4-x^2$};
        \node[below left] {$0$};
        \draw[fill] (-1,0.3) circle (0.04);
        \draw[fill] (4,-1.2) circle (0.04);
        \draw[dashed] (-1,0.3) -- (-1,0) node[below] {$-1$};
        \draw[dashed] (4,-1.2) -- (4,0) node[above] {$4$};
    \end{tikzpicture}
\end{center}

\tcblower

\vspace{-0.2cm}

\begin{align*}
    &\displaystyle{\int_{-1}^4 \left(4-x^2-(-3x)\right)}\,dx&&\color{blue!70!black!70!}\longleftarrow \text{Upper}-\text{Lower}\\[0.5em]
    &\displaystyle{\int_{-1}^4 \left(4-x^2+3x\right)}\,dx&&\color{blue!70!black!70!}\longleftarrow \text{Simplify}\\[0.5em]
    =\;&\left[4x-\frac{x^3}{3}+\frac{3x^2}{2}\right]_{-1}^4&&\color{blue!70!black!70!}\longleftarrow \text{Integrate}\\[0.5em]
    =\;&\left(4(4)-\frac{(4)^3}{3}+\frac{3(4)^2}{2}\right)-\left(4(-1)-\frac{(-1)^3}{3}+\frac{3(-1)^2}{2}\right)&&\color{blue!70!black!70!}\longleftarrow \text{Substitute}\\[0.5em]
    =\;&\frac{56}{3}-\left(-\frac{13}{6}\right)&&\color{blue!70!black!70!}\longleftarrow \text{Evaluate}\\[0.5em]
    =\;&\frac{125}{6}\text{ square units}
\end{align*}

\end{tcolorbox}

\newpage

\section*{Review Exercise}\label{review-exercise}
\addcontentsline{toc}{section}{Review Exercise}

\chapterfont{\color{white}}

\chapter{Addition Formulae}\label{addition-formulae}

\vspace{-12cm}
\begin{center}
\begin{tikzpicture}
\draw[white] (10,5) circle (0.01);
\draw[DarkOrchid,fill=DarkOrchid] (0,0) rectangle (18.2,0.2);
\draw[DarkOrchid,fill=DarkOrchid] (8,1.8) rectangle (18.2,0.2);
\node[white] at (13.1,1) {\Huge{\textsc{Addition Formulae}}};
\node at (4,1) {\Large{\textsc{Chapter 12}}};
\end{tikzpicture}
\end{center}

\chapterfont{\color{white}}

\chapter{The Circle}\label{the-circle}

\vspace{-12cm}
\begin{center}
\begin{tikzpicture}
\draw[white] (10,5) circle (0.01);
\draw[DarkOrchid,fill=DarkOrchid] (0,0) rectangle (18.2,0.2);
\draw[DarkOrchid,fill=DarkOrchid] (8,1.8) rectangle (18.2,0.2);
\node[white] at (13.1,1) {\Huge{\textsc{The Circle}}};
\node at (4,1) {\Large{\textsc{Chapter 13}}};
\end{tikzpicture}
\end{center}

\chapterfont{\color{white}}

\chapter{The Wave Function}\label{the-wave-function}

\vspace{-12cm}

\begin{center}
    \begin{tikzpicture}
        \draw[white] (10,5) circle (0.01);
        \draw[DarkOrchid,fill=DarkOrchid] (0,0) rectangle (18.2,0.2);
        \draw[DarkOrchid,fill=DarkOrchid] (6,1.8) rectangle (18.2,0.2);
        \node[white] at (12.1,1) {\Huge{\textsc{The Wave Function}}};
        \node at (3,1) {\Large{\textsc{Chapter 14}}};
    \end{tikzpicture}
\end{center}

In this chapter, the sums and differences of \emph{equal-angled} trigonometric operations will be explored.

\[\text{e.g. }3\sin{x}\degree+4\cos{x}\degree\]

The graphs of basic trigonometric functions such as \(y=3\sin{x}\degree\) and \(y=4\cos{x}\degree\) should be familiar:

\begin{center}
    \begin{tikzpicture}
        \draw[-stealth] (0,0) -- (14,0) node[below] {$x$};
        \draw[-stealth] (0,-3) -- (0,3) node[left] {$y$};
        \draw[smooth,domain=0:2*3.141592,xscale=2] plot (\x,{0.5*3*sin(\x r)});
        \draw[smooth,domain=0:2*3.141592,xscale=2] plot (\x,{0.5*4*cos(\x r)});
        \node[above] at (4*3.141592,0) {$360\degree$};
        \foreach \y in {-5,-4,...,5}{
        \draw (0,0.5*\y) -- (-0.1,0.5*\y) node[left] {\y};
}
        \node[below] at (3.5*3.141592,-1.35) {$y=3\sin{x}\degree$};
        \node[above] at (4*3.141592,2) {$y=4\cos{x}\degree$};
    \end{tikzpicture}
\end{center}

The graph of \(\color{DarkOrchid}y=3\sin{x}\degree+4\cos{x}\degree\) is the same as that of \(\color{DarkOrchid}y=5\cos{(x-36.9)}\degree\).

\begin{center}
    \begin{tikzpicture}
        \draw[-stealth] (0,0) -- (14,0) node[below] {$x$};
        \draw[-stealth] (0,-3) -- (0,3) node[left] {$y$};
        \draw[black!50,smooth,domain=0:2*3.141592,xscale=2] plot (\x,{0.5*3*sin(\x r)});
        \draw[black!50,smooth,domain=0:2*3.141592,xscale=2] plot (\x,{0.5*4*cos(\x r)});
        \draw[smooth,very thick,DarkOrchid,domain=0:2*3.141592,xscale=2] plot (\x,{0.5*4*cos(\x r)+0.5*3*sin(\x r)});
        \node[above] at (4*3.141592,0) {$360\degree$};
        \foreach \y in {-5,-4,...,5}{
        \draw (0,0.5*\y) -- (-0.1,0.5*\y) node[left] {\y};
}
        \node[black!50,below] at (3.5*3.141592,-1.35) {$y=3\sin{x}\degree$};
        \node[black!50,above] at (4*3.141592,2) {$y=4\cos{x}\degree$};
        \node[above,DarkOrchid] at (1.3*3.141592,2.5) {$y=3\sin{x}\degree+4\cos{x}\degree=5\cos{(x-36.9)}\degree$};
        \draw[dashed,DarkOrchid] (2*0.644,2.5) -- (2*0.644,0) node[below] {$36.9\degree$};
    \end{tikzpicture}
\end{center}

Determining that \(3\sin{x}\degree+4\cos{x}\degree=5\cos{(x-36.9)}\degree\) allows the maximum values and minimum values of \(3\sin{x}\degree+4\cos{x}\degree\) to be determined and the values of \(x\) which produce them, and allow equations such as \(3\sin{x}\degree+4\cos{x}\degree=2\) to be solved.

\emph{Any} trigonometric expression \(k_1\sin{x}\pm k_2\cos{x}\) can be written as either \(k\sin{(x\pm a)}\) or \(k\cos{(x\pm a)}\).

This chapter will cover how to do this and explore the various applications of this property.

\pagebreak

\section{\texorpdfstring{The Wave Function using \(k\cos{(x-a)}\)}{The Wave Function using k\textbackslash cos\{(x-a)\}}}\label{the-wave-function-using-kcosx-a}

\vspace{-0.5cm}

Whilst any of \(k\sin{(x\pm a)}\) or \(k\cos{(x\pm a)}\) may be used in general, any Higher exam question is likely to specify a particular form to use. The simplest is often \(k\cos{(x-a)}\).

To express \(3\sin{x}\degree+4\cos{x}\degree\) in the form \(k\cos{(x-a)}\degree\), two key \emph{trigonometric identities} are needed:

\[\sin^2{x}+\cos^2{x}=1\hspace{2cm}\text{and}\hspace{2cm}\tan{x}=\frac{\sin{x}}{\cos{x}}\]

First, \(k\cos{(x-a)}\degree\) can be expanded using the the formula \(\cos{(A\pm B)}=\cos{A}\cos{B}\mp \sin{A}\sin{B}\):

\[k\cos{(x-a)}\degree=k\cos{x}\degree\cos{a}\degree+k\sin{x}\degree\sin{a}\degree\]

It is required that this expansion is equal to the original expression:

\[3\underline{\sin{x}\degree}+4\doubleunderline{\cos{x}\degree}=k\doubleunderline{\cos{x}\degree}\cos{a}\degree+k\underline{\sin{x}\degree}\sin{a}\degree\]

Equating the coefficients of \(\underline{\sin{x}\degree}\) and \(\doubleunderline{\cos{x}\degree}\) gives a set of simultaneous equations in \(k\) and \(a\):

\begin{align*}
    k\sin{a}\degree&=3&&\color{DarkOrchid}\longleftarrow \text{Coefficients of }\underline{\sin{x}\degree}\\
    k\cos{a}\degree&=4&&\color{DarkOrchid}\longleftarrow \text{Coefficients of }\doubleunderline{\cos{x}\degree}
\end{align*}

Squaring both sides of each equation and adding allows \(k\) to be determined using \(\sin^2{x}+\cos^2{x}=1\):

\begin{align*}
    k^2\sin^2{a}\degree+k^2\cos^2{a}\degree&=3^2+4^2\\
    k^2\left(\sin^2{x}\degree+\cos^2{x}\degree\right)&=3^2+4^2\\
    k^2(1)&=3^2+4^2\\
    \color{DarkOrchid}k&\color{DarkOrchid}=\sqrt{3^2+4^2}\\
    k&=\sqrt{25}\\
    k&=5
\end{align*}

Dividing \(k\sin{a}\degree\) by \(k\cos{a}\degree\) allows \(a\) to be calculated, noting \(k\sin{a}\degree\) and \(k\cos{a}\degree\) are both positive:

\vspace{2.8cm}
\hspace{14cm}
\begin{tikzpicture}[remember picture, overlay, scale=0.5]
        \draw (-1.5,0) -- (1.5,0);
        \draw (0,-1) -- (0,1);
        \node at (0.7,0.5) {A};
        \node at (-0.7,0.5) {S};
        \node at (-0.7,-0.5) {T};
        \node at (0.7,-0.5) {C};
        \node[DarkOrchid,above right] at (0.7,0.65) {\scriptsize{$a\degree$}};
        \node[DarkOrchid,right] at (0.7,0.5) {\ding{51}\ding{51}};
        \node[DarkOrchid,left] at (-0.7,0.5) {\ding{51}};
        \node[DarkOrchid,right] at (0.7,-0.5) {\ding{51}};
\end{tikzpicture}

\vspace{-3.8cm}

\begin{align*}
    \color{DarkOrchid}\frac{k\sin{a}\degree}{k\cos{a}\degree}&\color{DarkOrchid}=\frac{3}{4}\\
    \tan{a}\degree&=\frac{3}{4}\\
    a\degree_{\text{acute}}&=\tan^{-1}{\left(\frac{3}{4}\right)}=36.9\degree\\
    a\degree&=36.9\degree
\end{align*}

Hence \(3\sin{x}\degree+4\cos{x}\degree=5\cos{(x-36.9)}\degree\). Here it has been assumed that \(k>0\) and \(0<a<360\).

\pagebreak

Some abbreviations of the working shown on the previous page are routinely permitted in the Higher exam. The following example demonstrates an appropriate level of detail in its solution.

\begin{tcolorbox}[title=Example,colback=DarkOrchid!1!, colframe=DarkOrchid]

Express $3\cos{x}-\sin{x}$ in the form $k\cos{(x-a)}$ where $k>0$ and $0<a<2\pi$.

\tcblower

\vspace{10cm}
\hspace{2.2cm}
\begin{tikzpicture}[remember picture, overlay, scale=0.5]
        \draw[white] (-8,0) -- (8,0);
        \draw (-1.5,0) -- (1.5,0);
        \draw (0,-1) -- (0,1);
        \node at (0.7,0.5) {A};
        \node at (-0.7,0.5) {S};
        \node at (-0.7,-0.5) {T};
        \node at (0.7,-0.5) {C};
        \node[DarkOrchid,below right] at (0.7,-0.65) {\scriptsize{$360\degree-\text{acute}\degree$}};
        \node[DarkOrchid,right] at (0.7,-0.5) {\ding{51}\ding{51}};
        \node[DarkOrchid,right] at (0.7,0.5) {\ding{51}};
        \node[DarkOrchid,left] at (-0.7,-0.5) {\ding{51}};
\end{tikzpicture}

\vspace{-4cm}
\hspace{8cm}
\begin{tikzpicture}[remember picture, overlay]
        \draw[thick,DarkOrchid,-stealth] (160:8) arc (160:200:8);
\end{tikzpicture}

\vspace{-7.5cm}
\begin{align*}
    3\cos{x}-\sin{x}&=k\cos{(x-a)}&&\\[0.5em]
    3\doubleunderline{\cos{x}}-\underline{\sin{x}}&=k\doubleunderline{\cos{x}}\cos{a}+k\underline{\sin{x}}\sin{a}&&\color{DarkOrchid}\longleftarrow \text{Expand }k\cos{(x-a)}\\[1.5em]
    k\sin{a}&=-1&&\color{DarkOrchid}\longleftarrow \text{Equate }\sin{x}\text{ coefficients}\\[0.5em]
    k\cos{a}&=3&&\color{DarkOrchid}\longleftarrow \text{Equate }\cos{x}\text{ coefficients}\\[1.5em]
    k&=\sqrt{(-1)^2+3^2}&&\color{DarkOrchid}\longleftarrow \text{Calculate }k\\[0.5em]
    &=\sqrt{10}\\[1em]
    \tan{a}&=\frac{-1}{3}&&\color{DarkOrchid}\longleftarrow \text{Find }\tan{a}\\[0.5em]
    \text{acute}\degree&=\tan^{-1}{\left(\frac{1}{3}\right)}=18.4\degree &&\color{DarkOrchid}\longleftarrow \text{Find acute angle first}\\[1.5em]
    &&&\color{DarkOrchid}\longleftarrow \text{Negative }\sin{a}\text{: quadrants T and C}\\[0.5em]
    &&&\color{DarkOrchid}\longleftarrow \text{Positive }\cos{a}\text{: quadrants A and C}\\[2em]
    a\degree &=341.6\degree&&\color{DarkOrchid}\longleftarrow \text{Using }360\degree-18.4\degree\text{ (Double-ticked quadrant)}\\[0.5em]
    a &=5.96&&\color{DarkOrchid}\longleftarrow \text{Convert to radians: }341.6\times\frac{\pi}{180}\\[1.5em]
    3\cos{x}-\sin{x}&=\sqrt{10}\cos{(x-5.96)}&&\color{DarkOrchid}\longleftarrow \text{State solution}
\end{align*}

\end{tcolorbox}

\vspace{-0.5cm}

\subsubsection*{Exercise 14.1}\label{exercise-14.1}
\addcontentsline{toc}{subsubsection}{Exercise 14.1}

\vspace{-0.3cm}

\begin{enumerate}
    \item Express each of the following in the form $k\cos{(x-a)}\degree$, where $k>0$ and $0<a<360$. \faCalculator
    \vspace{-0.2cm}
    \begin{enumerate}
        \begin{multicols}{3}
            \item $5\cos{x}\degree+12\sin{x}\degree$
            \item $4\sin{x}\degree+5\cos{x}\degree$
            \item $6\cos{x}\degree+\sin{x}\degree$
        \end{multicols}
    \end{enumerate}
    \item Express each of the following in the form $k\cos{(x-a)}$, where $k>0$ and $0<a<2\pi$. \faCalculator
    \vspace{-0.2cm}
    \begin{enumerate}
        \begin{multicols}{3}
            \item $8\sin{x}-6\cos{x}$
            \item $\cos{x}-3\sin{x}$
            \item $-2\sin{x}-\cos{x}$
        \end{multicols}
    \end{enumerate}
    \item Express each of the following in the form $k\cos{(x-a)}\degree$, where $k>0$ and $0<a<360$.
    \vspace{-0.2cm}
    \begin{enumerate}
        \begin{multicols}{3}
            \item $\sin{x}\degree+\cos{x}\degree$
            \item $\cos{x}\degree-\sqrt{3}\sin{x}\degree$
            \item $-\sin{x}\degree-\sqrt{3}\cos{x}\degree$
        \end{multicols}
    \end{enumerate}
\end{enumerate}

\pagebreak

\section{Other Forms of the The Wave Function}\label{other-forms-of-the-the-wave-function}

\vspace{-0.5cm}

As well as \(k\cos{(x-a)}\), three other forms may be used, shown below including their expansions:

\vspace{-1.0cm}

\begin{align*}
    k\cos{(x+a)}&\color{DarkOrchid}=k\cos{x}\cos{a}-k\sin{x}\sin{a}\\
    k\sin{(x+a)}&\color{DarkOrchid}=k\sin{x}\cos{a}+k\sin{x}\cos{a}\\
    k\sin{(x-a)}&\color{DarkOrchid}=k\sin{x}\cos{a}-k\sin{x}\cos{a}
\end{align*}

\vspace{-0.3cm}

\begin{tcolorbox}[title=Example,colback=DarkOrchid!1!, colframe=DarkOrchid]

Express $12\cos{x}\degree-5\sin{x}\degree$ in the form $k\sin{(x-a)}\degree$ where $k>0$ and $0<a<360$.

\tcblower

\vspace{9.7cm}
\hspace{3.2cm}
\begin{tikzpicture}[remember picture, overlay, scale=0.5]
        \draw[white] (-8,0) -- (8,0);
        \draw (-1.5,0) -- (1.5,0);
        \draw (0,-1) -- (0,1);
        \node at (0.7,0.5) {A};
        \node at (-0.7,0.5) {S};
        \node at (-0.7,-0.5) {T};
        \node at (0.7,-0.5) {C};
        \node[DarkOrchid,below left] at (-0.7,-0.65) {\scriptsize{$180\degree+\text{acute}\degree$}};
        \node[DarkOrchid,left] at (-0.7,-0.5) {\ding{51}\ding{51}};
        \node[DarkOrchid,right] at (0.7,-0.5) {\ding{51}};
        \node[DarkOrchid,left] at (-0.7,0.5) {\ding{51}};
\end{tikzpicture}

\vspace{-3.7cm}
\hspace{8.5cm}
\begin{tikzpicture}[remember picture, overlay]
        \draw[thick,DarkOrchid,-stealth] (160:8) arc (160:200:8);
\end{tikzpicture}

\vspace{-7.5cm}
\begin{align*}
    12\cos{x}\degree-5\sin{x}\degree&=k\sin{(x-a)}\degree&&\\[0.5em]
    12\doubleunderline{\cos{x}\degree}-5\underline{\sin{x}\degree}&=k\underline{\sin{x}\degree}\cos{a}\degree-k\doubleunderline{\cos{x}\degree}\sin{a}\degree&&\color{DarkOrchid}\longleftarrow \text{Expand }k\sin{(x-a)}\degree\\[1.5em]
    -k\sin{a}\degree&=12\implies k\sin{a}\degree=-12&&\color{DarkOrchid}\longleftarrow \text{Equate }\sin{x}\degree\text{ coefficients}\\[0.5em]
    k\cos{a}\degree&=-5&&\color{DarkOrchid}\longleftarrow \text{Equate }\cos{x}\degree\text{ coefficients}\\[1.5em]
    k&=\sqrt{(-12)^2+(-5)^2}&&\color{DarkOrchid}\longleftarrow \text{Calculate }k\\[0.5em]
    &=13\\[1.5em]
    \tan{a}\degree&=\frac{-12}{-5}=\frac{12}{5}&&\color{DarkOrchid}\longleftarrow \text{Find }\tan{a}\degree\\[0.5em]
    \text{acute}\degree&=\tan^{-1}{\left(\frac{12}{5}\right)}=67.4\degree &&\color{DarkOrchid}\longleftarrow \text{Find acute angle first}\\[0.8em]
    &&&\color{DarkOrchid}\longleftarrow \text{Negative }\sin{a}\degree\text{: quadrants T and C}\\[0.5em]
    &&&\color{DarkOrchid}\longleftarrow \text{Negative }\cos{a}\degree\text{: quadrants S and T}\\[1.5em]
    a\degree &=247.4\degree&&\color{DarkOrchid}\longleftarrow \text{Using }180\degree+67.4\degree\text{ (Double-ticked)}\\[1em]
    12\cos{x}\degree-5\sin{x}\degree&=13\sin{(x-247.4)}\degree&&\color{DarkOrchid}\longleftarrow \text{State solution}
\end{align*}

\end{tcolorbox}

\vspace{-0.5cm}

\subsubsection*{Exercise 14.2}\label{exercise-14.2}
\addcontentsline{toc}{subsubsection}{Exercise 14.2}

\vspace{-0.3cm}

\begin{enumerate}
    \item Express each of the following in the form $k\sin{(x+a)}\degree$, where $k>0$ and $0<a<360$. \faCalculator
    \vspace{-0.2cm}
    \begin{enumerate}
        \begin{multicols}{3}
            \item $3\cos{x}\degree+2\sin{x}\degree$
            \item $4\sin{x}\degree+3\cos{x}\degree$
            \item $7\cos{x}\degree-\sin{x}\degree$
        \end{multicols}
    \end{enumerate}
    \item Express each of the following in the form $k\cos{(x+a)}$, where $k>0$ and $0<a<2\pi$. \faCalculator
    \vspace{-0.2cm}
    \begin{enumerate}
        \begin{multicols}{3}
            \item $3\sin{x}+8\cos{x}$
            \item $-3\cos{x}-4\sin{x}$
            \item $-6\sin{x}+2\cos{x}$
        \end{multicols}
    \end{enumerate}
    \item Express each of the following in the form $k\sin{(x-a)}$, where $k>0$ and $0<a<2\pi$.
    \vspace{-0.2cm}
    \begin{enumerate}
        \begin{multicols}{3}
            \item $\sqrt{3}\sin{x}+\cos{x}$
            \item $\cos{x}-\sin{x}$
            \item $2\sqrt{3}\sin{x}-2\cos{x}$
        \end{multicols}
    \end{enumerate}
\end{enumerate}

\pagebreak

\section{Maximum and Minimum Values using the Wave Function}\label{maximum-and-minimum-values-using-the-wave-function}

\vspace{-0.5cm}

The graphs of \(y=\sin{x}\degree\) and \(y=\cos{x}\degree\), including their turning points, should already be familiar:

\begin{multicols}{2}
    \begin{center}
        \begin{tikzpicture}
            \draw[-stealth] (0,0) -- (7,0) node[below] {$x$};
            \draw[-stealth] (0,-1.4) -- (0,1.4) node[left] {$y$};
            \draw[smooth,domain=0:2*3.141592] plot (\x,{sin(\x r)});
            \node[above] at (2*3.141592,0) {$360$};
            \node[left] at (0,1) {$1$};
            \node[left] at (0,0) {$0$};
            \node[left] at (0,-1) {$-1$};
            \node at (3.3,1) {$y=\sin{x}\degree$};
            \foreach \x in {0.5,1,1.5,2}{
            \draw (3.141592*\x,0) -- (3.141592*\x,-0.1);
}
        \end{tikzpicture}
    \end{center}
    \begin{center}
        \begin{tikzpicture}
            \draw[-stealth] (0,0) -- (7,0) node[below] {$x$};
            \draw[-stealth] (0,-1.4) -- (0,1.4) node[left] {$y$};
            \draw[smooth,domain=0:2*3.141592] plot (\x,{cos(\x r)});
            \node[below] at (2*3.141592,0) {$360$};
            \node[left] at (0,1) {$1$};
            \node[left] at (0,0) {$0$};
            \node[left] at (0,-1) {$-1$};
            \node at (3.3,1) {$y=\cos{x}\degree$};
            \draw[dashed] (2*3.141592,1) -- (2*3.141592,0);
            \foreach \x in {0.5,1,1.5,2}{
            \draw (3.141592*\x,0) -- (3.141592*\x,-0.1);
}
        \end{tikzpicture}
    \end{center}
\end{multicols}

Graphs of the form \(k\sin{(x\pm a)}\) and \(k\cos{(x\pm a)}\) have maximum/minimum values of \(\pm k\), and their \emph{horizontal translation} is described by \(\pm a\) following the rules covered in the Graph Transformations chapter.

\begin{tcolorbox}[title=Example,colback=DarkOrchid!1!, colframe=DarkOrchid]

Given that $3\sin{x}\degree+4\cos{x}\degree$ can be expressed as $5\cos{(x-36.9)}\degree$, state the minimum value of $f(x)=3\sin{x}\degree+4\cos{x}\degree$ and the value of $x$ at which it occurs for $0<x<360$.

\tcblower

\textcolor{DarkOrchid}{Sketch the graph of $y=5\cos{x}\degree$ translated $36.9\degree$ to the right:}

\begin{center}
        \begin{tikzpicture}
            \draw[-stealth] (0,0) -- (7,0) node[below] {$x$};
            \draw[-stealth] (0,-1.6) -- (0,2) node[left] {$y$};
            \draw[thin,black!50,smooth,domain=0:2*3.141592] plot (\x,{0.3*5*cos(\x r)});
            \draw[thick,DarkOrchid,smooth,domain=0:2*3.141592] plot (\x,{0.3*3*sin(\x r)+0.3*4*cos(\x r)});
            \draw[thick,dashed,DarkOrchid,smooth,domain=2*3.141592:6.9272] plot (\x,{0.3*3*sin(\x r)+0.3*4*cos(\x r)});
            \node[black!50] at (2*3.141592,1.8) {$y=5\cos{x}\degree$};
            \node[right,DarkOrchid] at (6.9272,1.5) {$y=5\cos{(x-36.9)}\degree$};
            \node[below] at (2*3.141592,0) {$360$};
            \node[below] at (3.141592,0) {$180$};
            \node[left] at (0,1.5) {$5$};
            \node[left] at (0,0) {$0$};
            \node[left] at (0,-1.5) {$-5$};
            \node at (3.3,1) {$y=\cos{x}\degree$};
            \draw[dashed,black!50] (2*3.141592,1) -- (2*3.141592,0);
            \foreach \x in {0.5,1,1.5,2}{
            \draw (3.141592*\x,0) -- (3.141592*\x,-0.1);
}
            \draw[stealth-,DarkOrchid] (3.8859,-1.6) -- (4.1859,-1.9) node[right] {$(216.9\degree,-5)$};
            \draw[stealth-,black!50] (3.041592,-1.6) -- (2.741592,-1.9) node[left] {$(180\degree,-5)$};
            \draw[DarkOrchid,fill=DarkOrchid] (3.7859,-1.5) circle (0.05);
            \draw[black!50,fill=black!50] (3.141592,-1.5) circle (0.05);
        \end{tikzpicture}
    \end{center}

\textcolor{DarkOrchid}{Use the graph to state the answer:}

\vspace{0.1cm}

Hence the minimum value of $f(x)$ is $-5$, which occurs when $x=216.9$.

\end{tcolorbox}

\vspace{-0.5cm}

\subsubsection*{Exercise 14.3}\label{exercise-14.3}
\addcontentsline{toc}{subsubsection}{Exercise 14.3}

\vspace{-0.3cm}

\begin{enumerate}
    \item Sketch each for $0<x<360$, showing the coordinates of any roots and turning points:
    \vspace{-0.2cm}
    \begin{enumerate}
        \begin{multicols}{3}
            \item $7\cos{(x-10)}\degree$
            \item $3\sin{(x-20)}\degree$
            \item $\sqrt{5}\cos{(x+40)}\degree$
        \end{multicols}
    \end{enumerate}
    \item Find the maximum value of each and the value(s) of $x$ for which they occur for $0<x<360$:
    \vspace{-0.2cm}
    \begin{enumerate}
        \begin{multicols}{3}
            \item $8\sin{(x-50)}\degree$
            \item $\sqrt{2}\cos{(x+27)}\degree$
            \item $7\sin{(x+42.3)}\degree$
        \end{multicols}
    \end{enumerate}
    \item Find the minimum value of each and the value(s) of $x$ for which they occur for $0<x<2\pi$:
    \vspace{-0.2cm}
    \begin{enumerate}
        \begin{multicols}{3}
            \item $3\cos{\left(x-\dfrac{\pi}{6}\right)}$
            \item $5\sin{\left(x+\dfrac{\pi}{3}\right)}+2$
            \item $-12\sin{\left(x+\dfrac{\pi}{4}\right)}$
        \end{multicols}
    \end{enumerate}
    \item
    \begin{enumerate}
        \item Express $6\sin{x}\degree-7\cos{x}\degree$ in the form $k\sin{(x-a)}\degree$ where $k>0$ and $0<a<360$. \faCalculator
        \item Hence state the coordinates of the turning points of $12\sin{x}\degree-14\cos{x}\degree$ for $0<x<360$.
    \end{enumerate}
\end{enumerate}

\pagebreak

\section{Solving Equations using the Wave Function}\label{solving-equations-using-the-wave-function}

\vspace{-0.5cm}

Solving an equation like \(3\sin{x}\degree+4\cos{x}\degree=2\) can be approached by using the wave function to rewrite it as \(5\cos{(x-36.9)}\degree=2\), before solving it in the manner covered in Chapter 6.

\begin{tcolorbox}[title=Example,colback=DarkOrchid!1!, colframe=DarkOrchid]

It can be shown that $5\sin{x}\degree-2\cos{x}\degree$ can be expressed as $\sqrt{29}\sin{(x-21.8)}\degree$.

Hence, solve the equation $5\sin{x}\degree-2\cos{x}\degree=4$ where $0<x<360$.

\tcblower

\vspace{5.6cm}
\hspace{7cm}
\begin{tikzpicture}[remember picture, overlay, scale=0.5]
        \draw[white] (-8,0) -- (8,0);
        \draw (-1.5,0) -- (1.5,0);
        \draw (0,-1) -- (0,1);
        \node at (0.7,0.5) {A};
        \node at (-0.7,0.5) {S};
        \node at (-0.7,-0.5) {T};
        \node at (0.7,-0.5) {C};
        \node[DarkOrchid,above right] at (0.7,0.65) {\scriptsize{$a\degree$}};
        \node[DarkOrchid,above left] at (-0.7,0.65) {\scriptsize{$180\degree-a\degree$}};
        \node[DarkOrchid,right] at (0.7,0.5) {\ding{51}};
        \node[DarkOrchid,left] at (-0.7,0.5) {\ding{51}};
\end{tikzpicture}

\vspace{-6.6cm}
\begin{align*}
    5\sin{x}\degree-2\cos{x}\degree&=4&&\\[0.5em]
    \sqrt{29}\sin{(x-21.8)}\degree&=4&&\color{DarkOrchid}\longleftarrow \text{Substitute the wave function form}\\[0.5em]
    \sin{(x-21.8)}\degree&=\frac{4}{\sqrt{29}}&&\color{DarkOrchid}\longleftarrow \text{Rearrange to }\sin{(\dots)}=\dots\\[0.5em]
    \color{DarkOrchid}a&\color{DarkOrchid}=\sin^{-1}{\left(\frac{4}{\sqrt{29}}\right)=48.0\degree}&&\color{DarkOrchid}\longleftarrow \text{Calculate acute angle}\\[2.5em]
    &&&\color{DarkOrchid}\longleftarrow \text{Positive }\sin{a}\degree\text{: quadrants A and S}\\[2.5em]
    x-21.8\degree&=48.0\degree,180\degree-48.0\degree&&\color{DarkOrchid}\longleftarrow \text{Apply ticked quadrants}\\[0.5em]
    x-21.8\degree&=48.0\degree,132.0\degree&&\\[0.5em]
    x\degree&=69.8\degree,153.8\degree&&\color{DarkOrchid}\longleftarrow \text{Add }21.8\degree\text{ to both sides}
\end{align*}

\end{tcolorbox}

Note that any solutions outwith the domain (often \(0<x<360\)) should have \(360\degree\) added or subtracted to bring it back within the domain, where possible.

\vspace{-0.5cm}

\subsubsection*{Exercise 14.4}\label{exercise-14.4}
\addcontentsline{toc}{subsubsection}{Exercise 14.4}

\vspace{-0.3cm}

\begin{enumerate}
    \item Solve each equation for $0<x<360$: \faCalculator
    \begin{enumerate}
        \begin{multicols}{3}
            \item $7\sin{(x-18)}\degree=4$
            \item $3\cos{(x+34.1)}\degree=-2$
            \item $\sqrt{5}\sin{(x-106)}\degree+1=0$
        \end{multicols}
    \end{enumerate}
    \item Given $6\sin{x}\degree-8\cos{x}\degree=10\sin{(x-53.1)}\degree$, solve $6\sin{x}\degree-8\cos{x}\degree=5$ where $0<x<360$.
    \item 
    \begin{enumerate}
        \item Express $\sqrt{3}\sin{x}\degree+\cos{x}\degree$ in the form $k\sin{(x-a)}\degree$ where $k>0$ and $0<a<360$. 
        \item Hence solve the equation $\sqrt{3}\sin{x}\degree+\cos{x}\degree=1$ where $0<x<360$.
    \end{enumerate}
    \item Solve each equation for $0<x<2\pi$: \faCalculator
    \begin{enumerate}
        \begin{multicols}{3}
            \item $4\sin{(x+0.31)}+2=1$
            \item $9\cos{(x+1.24)}=5$
            \item $2\sqrt{3}\sin{(x-0.82)}=\sqrt{5}$
        \end{multicols}
    \end{enumerate}
    \item 
    \begin{enumerate}
        \item Express $3\cos{x}+2\c{sinx}$ in the form $k\cos{(x+a)}$ where $k>0$ and $0<a<2\pi$. \faCalculator
        \item Hence solve the equation $2+6\sin{x}+4\cos{x}=5$ where $0<x<2\pi$. \faCalculator
    \end{enumerate}
\end{enumerate}

\pagebreak

\section{Multiple Angles and Different Variables}\label{multiple-angles-and-different-variables}

\vspace{-0.5cm}

The techniques covered in this chapter can be applied to trigonometric expressions beyond those containing only \(\sin{x}\) and \(\cos{x}\); they work for any sum or difference of \emph{equal-angled} trigonometric operations.

\vspace{-0.2cm}

\[\text{e.g. }\qquad 3\sin{t}\degree+4\cos{t}\degree \qquad \text{ or } \qquad 2\sin{2x}-5\cos{2x}\]

\begin{tcolorbox}[title=Example,colback=DarkOrchid!1!, colframe=DarkOrchid]

Express $5\cos{2t}\degree-3\sin{2t}\degree$ in the form $k\sin{(2t+a)}\degree$ where $k>0$ and $0<a<360$.

\tcblower

\vspace{10cm}
\hspace{4cm}
\begin{tikzpicture}[remember picture, overlay, scale=0.5]
        \draw[white] (-8,0) -- (8,0);
        \draw (-1.5,0) -- (1.5,0);
        \draw (0,-1) -- (0,1);
        \node at (0.7,0.5) {A};
        \node at (-0.7,0.5) {S};
        \node at (-0.7,-0.5) {T};
        \node at (0.7,-0.5) {C};
        \node[DarkOrchid,above left] at (-0.7,0.65) {\scriptsize{$180\degree-\text{acute}\degree$}};
        \node[DarkOrchid,left] at (-0.7,0.5) {\ding{51}\ding{51}};
        \node[DarkOrchid,right] at (0.7,0.5) {\ding{51}};
        \node[DarkOrchid,left] at (-0.7,-0.5) {\ding{51}};
\end{tikzpicture}

\vspace{-4cm}
\hspace{8.3cm}
\begin{tikzpicture}[remember picture, overlay]
        \draw[thick,DarkOrchid,-stealth] (160:8) arc (160:200:8);
\end{tikzpicture}

\vspace{-7.6cm}
\begin{align*}
    5\cos{2t}\degree-3\sin{2t}\degree&=k\sin{(2t+a)}\degree&&\\[0.5em]
    5\doubleunderline{\cos{2t}\degree}-3\underline{\sin{2t}\degree}&=k\underline{\sin{2t}\degree}\cos{a}\degree+k\doubleunderline{\cos{2t}\degree}\sin{a}\degree&&\color{DarkOrchid}\longleftarrow \text{Expand }k\sin{(2t+a)}\degree\\[1.5em]
    k\sin{a}\degree&=5&&\color{DarkOrchid}\longleftarrow \text{Equate }\sin{2t}\degree\text{ coefficients}\\[0.5em]
    k\cos{a}\degree&=-3&&\color{DarkOrchid}\longleftarrow \text{Equate }\cos{2t}\degree\text{ coefficients}\\[1.5em]
    k&=\sqrt{(5)^2+(-3)^2}&&\color{DarkOrchid}\longleftarrow \text{Calculate }k\\[0.5em]
    &=\sqrt{34}\\[1.5em]
    \tan{a}\degree&=\frac{5}{-3}=-\frac{5}{3}&&\color{DarkOrchid}\longleftarrow \text{Find }\tan{a}\degree\\[0.5em]
    \text{acute}\degree&=\tan^{-1}{\left(\frac{5}{3}\right)}=59.0\degree &&\color{DarkOrchid}\longleftarrow \text{Find acute angle first}\\[0.8em]
    &&&\color{DarkOrchid}\longleftarrow \text{Positive }\sin{a}\degree\text{: quadrants A and S}\\[0.5em]
    &&&\color{DarkOrchid}\longleftarrow \text{Negative }\cos{a}\degree\text{: quadrants S and T}\\[1.5em]
    a\degree &=121.0\degree&&\color{DarkOrchid}\longleftarrow \text{Using }180\degree-59.0\degree\text{ (Double-ticked)}\\[1em]
    5\cos{2t}\degree-3\sin{2t}\degree&=\sqrt{34}\sin{(2t+121.0)}\degree&&\color{DarkOrchid}\longleftarrow \text{State solution}
\end{align*}

\end{tcolorbox}

\vspace{-0.5cm}

\subsubsection*{Exercise 14.5}\label{exercise-14.5}
\addcontentsline{toc}{subsubsection}{Exercise 14.5}

\vspace{-0.3cm}

\begin{enumerate}
    \item Express $4\cos{t}\degree-3\sin{t}\degree$ in the form $k\sin{(t-a)}\degree$, where $k>0$ and $0<a<360$. \faCalculator
    \item Express $2\sin{2x}-\cos{2x}$ in the form $k\cos{(2x-a)}$ where $k>0$ and $0<a<2\pi$. \faCalculator
    \item 
    \begin{enumerate}
        \item Express $12\cos{t}\degree+5\sin{t}\degree$ in the form $k\sin{(t+a)}\degree$, where $k>0$ and $0<a<360$. \faCalculator
        \item Hence state:
        \begin{enumerate}
            \item The maximum value of the function $f(x)=12\cos{t}\degree+5\sin{t}\degree$, $0<t<360$.
            \item The value(s) of $t$ for which it occurs.
        \end{enumerate}
    \end{enumerate}
    \item 
    \begin{enumerate}
        \item Express $\sin{2x}-\sqrt{3}\cos{2x}$ in the form $k\cos{(2x-a)}$ where $k>0$, $0<a<2\pi$.
        \item Hence solve $\sin{2x}-\sqrt{3}\cos{2x}-1=0$, $0<x<2\pi$.
        \item Sketch $y=-\sin{2x}+\sqrt{3}\cos{2x}-1=0$ for $0\leqslant x \leqslant 2\pi$.
    \end{enumerate}
\end{enumerate}

\pagebreak

\section*{Wave Function Review Exercise}\label{wave-function-review-exercise}
\addcontentsline{toc}{section}{Wave Function Review Exercise}

\begin{enumerate}
    \item Express $8\sin{x}\degree+7\cos{x}\degree$ in the form $k\sin{(x-a)}\degree$ where $k>0$ and $0<a<360$. \faCalculator\\[0.5em]
    \item Express $\sqrt{5}\sin{x}+\cos{x}$ in the form $k\cos{(x-a)}$ where $k>0$ and $0<a<2\pi$. \faCalculator\\[0.5em]
    \item Express $\sqrt{3}\cos{t}\degree-\sin{t}\degree$ in the form $k\sin{(t+a)}\degree$ where $k>0$ and $0<a<360$.\\[0.5em]
    \item Part of the graphs of $y=3\cos{x}\degree-5\sin{x}\degree$ and $y=-4$ are shown in the diagram below: \faCalculator
    \begin{center}
        \begin{tikzpicture}[scale=1.5]
            \draw[-stealth] (-0.5,0) -- (7.5,0) node[below] {$x$};
            \draw[-stealth] (0,-1.5) -- (0,1.5) node[left] {$y$};
            \node[below left] {O};
            \draw[thick,smooth,domain=0:2*3.141592] plot (\x,{0.6*cos(\x r)-1.0*sin(\x r)}) node[right] {$y=3\cos{x}\degree-5\sin{x}\degree$};
            \draw[dashed] (2*3.141592,0.6) -- (2*3.141592,0) node[below] {$360\degree$};
            \draw[thick] (0,-0.8) -- (2*3.141592,-0.8) node[right] {$y=-4$};
            \draw[fill] (1.29639,-0.8) circle (0.04) node[below left] {P};
            \draw[fill] (2.92604,-0.8) circle (0.04) node[below right] {Q};
        \end{tikzpicture}
    \end{center}
    Points P and Q are points of intersection.\\
    \begin{enumerate}
        \item Express $y=3\cos{x}\degree-5\sin{x}\degree$ in the form $k\cos{(x+a)}\degree$ where $k>0$ and $0<a<360$.\\[0.1em]
        \item Hence determine the coordinates of P and Q.\\[0.5em]
    \end{enumerate}
    \item 
    \begin{enumerate}
        \item Express $2\sin{x}\degree-4\cos{x}\degree$ in the form $k\sin{(x-a)}\degree$ where $k>0$ and $0<x<360$. \faCalculator\\[0.1em]
        \item Hence sketch the graph of $y=2\sin{x}\degree-4\cos{x}\degree$ for $0<x<360$.\\[0.5em]
    \end{enumerate}
    \item 
    \begin{enumerate}
        \item Express $\cos{x}+\sqrt{3}\sin{x}$ in the form $k\cos{(x-a)}$ where $k>0$ and $0<a<2\pi$.\\[0.1em]
        \item Hence sketch the graph of $y=2\cos{x}+2\sqrt{3}\sin{x}$ for $0<x<2\pi$.\\[0.5em]
    \end{enumerate}
    \item 
    \begin{enumerate}
        \item Express $6\cos{t}-3\sin{t}$ in the form $k\cos{(t+a)}$ where $k>0$ and $0<a<2\pi$. \faCalculator\\[0.1em]
        \item Hence solve $2\cos{t}-\sin{t}+2=1$ where $0<t<2\pi$. \faCalculator\\[0.5em]
    \end{enumerate}
    \item 
    \begin{enumerate}
        \item Express $\sin{2x}\degree-\cos{2x}\degree$ in the form $k\sin{(2x-a)}$ where $k>0$ and $0<a<360$.\\[0.1em]
        \item Hence solve the equation $\sin{2x}\degree=\cos{2x}\degree$ where $0<x<360$.
    \end{enumerate}
\end{enumerate}

\pagebreak

\chapterfont{\color{white}}

\chapter{Logs and Exponentials}\label{logs-and-exponentials}

\vspace{-12cm}
\begin{center}
\begin{tikzpicture}
\draw[white] (10,5) circle (0.01);
\draw[PineGreen,fill=PineGreen] (0,0) rectangle (18.2,0.2);
\draw[PineGreen,fill=PineGreen] (6,1.8) rectangle (18.2,0.2);
\node[white] at (12.1,1) {\Huge{\textsc{Logs and Exponentials}}};
\node at (3,1) {\Large{\textsc{Chapter 15}}};
\end{tikzpicture}
\end{center}

Each part of an operation of the form \(a^n\) can be described using the following terminology:

\begin{center}
    \begin{tikzpicture}
        \node at (0,0) {\large{$a^n$}};
        \draw[stealth-,PineGreen,thick] (0.3,0.2) --++(1,0.5) node[right] {"power" or "exponent"};
        \draw[stealth-,PineGreen,thick] (-0.32,0) --++(-1,0.2) node[left] {"base"};
    \end{tikzpicture}
\end{center}

\emph{Power functions} take the form \(f(x)=x^n\), with \(x\) as the \textcolor{PineGreen}{\textit{base}} and a constant power, \(n\).

\[\text{e.g.}\quad f(x)=x^3\]

\emph{Exponential functions} take the form \(f(x)=a^x\), with \(x\) as the \textcolor{PineGreen}{\textit{exponent}} and a constant base, \(a>0,a\ne 1\).

\[\text{e.g.}\quad f(x)=3^x\]

An example of an exponential function in everyday life is that of something \emph{appreciating} by a percentage of its value, such as an antique vase of value £4000 increasing by 20\% each year. Its value after \(0\) years, \(1\) years, \(2\) years, and so on, can be calculated as follows:

\begin{align*}
    \color{black!25}4000\times 1.2^0\;=\;&4000 && \color{PineGreen}\longleftarrow\text{After 0 years}\\[0.5em]
    4000\times 1.2^{\color{PineGreen}1}\color{black}\;=\;&4800 && \color{PineGreen}\longleftarrow\text{After 1 year}\\[0.5em]
    4000\times 1.2^{\color{PineGreen}2}\color{black}\;=\;&5760 && \color{PineGreen}\longleftarrow\text{After 2 years}\\[0.5em]
    4000\times 1.2^{\color{PineGreen}3}\color{black}\;=\;&6912 && \color{PineGreen}\longleftarrow\text{After 3 years}\\[0.5em]
    4000\times 1.2^{\color{PineGreen}4}\color{black}\;=\;&8294.40 && \color{PineGreen}\longleftarrow\text{After 4 years}
\end{align*}

The function to describe its value \(V\) after \(\color{PineGreen}x\) years is given by: \(V(\color{PineGreen}x\color{black})=4000\times1.2^{\color{PineGreen}x}\)

One advantage of defining this function is the ability to calculate the value at times other that after whole years. For example, the value after \textcolor{PineGreen}{three and a half years} can be calculated as:

\[V(\color{PineGreen}3.5\color{black})=4000\times1.2^{\color{PineGreen}3.5}=7571.72\]

This chapter will introduce a range of skills required when working with exponential functions.

\pagebreak

\section{Graphs of Exponential Functions}\label{graphs-of-exponential-functions}

\vspace{-0.5cm}

\setlength{\columnsep}{30pt}

\begin{multicols}{2}

Where $a>1$:

$y=a^x$ is \textit{strictly increasing} on $x\in\mathbb{R}$:

\vspace{-0.5cm}
\begin{center}
    \begin{tikzpicture}[scale=0.8]
        \draw[-stealth] (-5,0) -- (5,0) node[below] {$x$};
        \draw[-stealth] (0,-1) -- (0,4.5) node[left] {$y$};
        \node[below left] {O};
        \draw[smooth,thick,PineGreen,domain=-5:5] plot (\x,{1.3^(\x)}) node[above left] {$y=a^x$};
        \draw[PineGreen,fill=PineGreen] (0,1) circle (0.05) node[above left] {$1$};
        \draw[PineGreen,fill=PineGreen] (1,1.3) circle (0.05) node[above] {$(1,a)$};
    \end{tikzpicture}
\end{center}

\vspace{-0.5cm}
This describes \textbf{exponential growth}.

\columnbreak

Where $0<a<1$:

$y=a^x$ is \textit{strictly decreasing} on $x\in\mathbb{R}$:

\vspace{-0.5cm}
\begin{center}
    \begin{tikzpicture}[scale=0.8]
        \draw[-stealth] (-5,0) -- (5,0) node[below] {$x$};
        \draw[-stealth] (0,-1) -- (0,4.5) node[left] {$y$};
        \node[below left] {O};
        \draw[smooth,thick,PineGreen,domain=-5:5] plot (\x,{0.76923^(\x)}) node[above] {$y=a^x$};
        \draw[PineGreen,fill=PineGreen] (0,1) circle (0.05) node[above left] {$1$};
        \draw[PineGreen,fill=PineGreen] (1,0.76923) circle (0.05) node[above] {$(1,a)$};
    \end{tikzpicture}
\end{center}

\vspace{-0.5cm}
This describes \textbf{exponential decay}.

\end{multicols}

\setlength{\columnsep}{10pt}

Since \(a^0=1\), any graph of the form \(y=a^x\) will pass through the point \((0,1)\) for all \(a\ne 0\).

Since \(a^1=a\), any graph of the form \(y=a^x\) will pass through the point \((1,a)\) for all \(a\).

Determining the equation of the graph of an exponential function can typically be achieved using substitution or consideration of graph transformations, along with knowledge of points \((0,1)\) and \((1,a)\).

\begin{tcolorbox}[title=Example,colback=PineGreen!2!, colframe=PineGreen]

The graph of $y=a^x+b$ is shown below. 

\vspace{-0.5cm}
\begin{center}
    \begin{tikzpicture}
        \draw[-stealth] (-5,0) -- (5,0) node[below] {$x$};
        \draw[-stealth] (0,-1) -- (0,4) node[left] {$y$};
        \node[below left] {O};
        \draw[smooth,thick,domain=-4:4,yscale=0.2] plot (\x,{2^(\x)+1}) node[above] {$y=a^x+b$};
        \draw[fill] (0,0.4) circle (0.05) node[above left] {$2$};
        \draw[fill] (2,1) circle (0.05) node[above left] {$(2,5)$};
    \end{tikzpicture}
\end{center}

\tcblower

\vspace{-0.5cm}
\begin{align*}
    2&=a^0+b && \color{PineGreen}\longleftarrow\text{Substitute }(0,2)\\[0.5em]
    2&=1+b &&\\[0.5em]
    1&=b && \color{PineGreen}\longleftarrow\text{Solve to obtain }b\\[1em]
    5&=a^2+1 && \color{PineGreen}\longleftarrow\text{Substitute }(2,5)\text{ and }b=1\\[0.5em]
    4&=a^2 &&\\[0.5em]
    2&=a && \color{PineGreen}\longleftarrow\text{Solve to obtain }a\\[1em]
    y&=2^x+1&&\color{PineGreen}\longleftarrow\text{State equation}
\end{align*}

\end{tcolorbox}

\pagebreak

\subsection*{Exercise 15.1}\label{exercise-15.1}
\addcontentsline{toc}{subsection}{Exercise 15.1}

\begin{enumerate}
    \item Find the equation of each exponential graphs using the form given.
    \begin{enumerate}
        \begin{multicols}{2}
            \item[(a)] $y=a^x$\\
            \begin{tikzpicture}[scale=0.8]
            \draw[-stealth] (-4,0) -- (4,0) node[below] {$x$};
            \draw[-stealth] (0,-2) -- (0,4) node[left] {$y$};
            \draw[smooth,thick,domain=-4:4,yscale=0.2] plot (\x,{2^(\x)+1});
            \draw[fill] (0,0.4) circle (0.05) node[above left] {$2$};
            \draw[fill] (2,1) circle (0.05) node[above left] {$(2,5)$};
            \end{tikzpicture}
            \item[(c)] $y=a^x$\\
            \begin{tikzpicture}[scale=0.8]
            \draw[-stealth] (-4,0) -- (4,0) node[below] {$x$};
            \draw[-stealth] (0,-2) -- (0,4) node[left] {$y$};
            \draw[smooth,thick,domain=-4:4,yscale=0.2] plot (\x,{2^(\x)+1});
            \draw[fill] (0,0.4) circle (0.05) node[above left] {$2$};
            \draw[fill] (2,1) circle (0.05) node[above left] {$(2,5)$};
            \end{tikzpicture}
            \item[(e)] $y=a^x$\\
            \begin{tikzpicture}[scale=0.8]
            \draw[-stealth] (-4,0) -- (4,0) node[below] {$x$};
            \draw[-stealth] (0,-2) -- (0,4) node[left] {$y$};
            \draw[smooth,thick,domain=-4:4,yscale=0.2] plot (\x,{2^(\x)+1});
            \draw[fill] (0,0.4) circle (0.05) node[above left] {$2$};
            \draw[fill] (2,1) circle (0.05) node[above left] {$(2,5)$};
            \end{tikzpicture}
            \item[(g)] $y=a^x$\\
            \begin{tikzpicture}[scale=0.8]
            \draw[-stealth] (-4,0) -- (4,0) node[below] {$x$};
            \draw[-stealth] (0,-2) -- (0,4) node[left] {$y$};
            \draw[smooth,thick,domain=-4:4,yscale=0.2] plot (\x,{2^(\x)+1});
            \draw[fill] (0,0.4) circle (0.05) node[above left] {$2$};
            \draw[fill] (2,1) circle (0.05) node[above left] {$(2,5)$};
            \end{tikzpicture}
            \item[(b)] $y=a^x$\\
            \begin{tikzpicture}[scale=0.8]
            \draw[-stealth] (-4,0) -- (4,0) node[below] {$x$};
            \draw[-stealth] (0,-2) -- (0,4) node[left] {$y$};
            \draw[smooth,thick,domain=-4:4,yscale=0.2] plot (\x,{2^(\x)+1});
            \draw[fill] (0,0.4) circle (0.05) node[above left] {$2$};
            \draw[fill] (2,1) circle (0.05) node[above left] {$(2,5)$};
            \end{tikzpicture}
            \item[(d)] $y=a^x$\\
            \begin{tikzpicture}[scale=0.8]
            \draw[-stealth] (-4,0) -- (4,0) node[below] {$x$};
            \draw[-stealth] (0,-2) -- (0,4) node[left] {$y$};
            \draw[smooth,thick,domain=-4:4,yscale=0.2] plot (\x,{2^(\x)+1});
            \draw[fill] (0,0.4) circle (0.05) node[above left] {$2$};
            \draw[fill] (2,1) circle (0.05) node[above left] {$(2,5)$};
            \end{tikzpicture}
            \item[(f)] $y=a^x$\\
            \begin{tikzpicture}[scale=0.8]
            \draw[-stealth] (-4,0) -- (4,0) node[below] {$x$};
            \draw[-stealth] (0,-2) -- (0,4) node[left] {$y$};
            \draw[smooth,thick,domain=-4:4,yscale=0.2] plot (\x,{2^(\x)+1});
            \draw[fill] (0,0.4) circle (0.05) node[above left] {$2$};
            \draw[fill] (2,1) circle (0.05) node[above left] {$(2,5)$};
            \end{tikzpicture}
            \item[(h)] $y=a^x$\\
            \begin{tikzpicture}[scale=0.8]
            \draw[-stealth] (-4,0) -- (4,0) node[below] {$x$};
            \draw[-stealth] (0,-2) -- (0,4) node[left] {$y$};
            \draw[smooth,thick,domain=-4:4,yscale=0.2] plot (\x,{2^(\x)+1});
            \draw[fill] (0,0.4) circle (0.05) node[above left] {$2$};
            \draw[fill] (2,1) circle (0.05) node[above left] {$(2,5)$};
            \end{tikzpicture}
        \end{multicols}
    \end{enumerate}
\end{enumerate}

\pagebreak

\section{Logarithms}\label{logarithms}

\chapterfont{\color{white}}

\chapter{Further Calculus}\label{further-calculus}

\vspace{-12cm}
\begin{center}
\begin{tikzpicture}
\draw[white] (10,5) circle (0.01);
\draw[DarkOrchid,fill=DarkOrchid] (0,0) rectangle (18.2,0.2);
\draw[DarkOrchid,fill=DarkOrchid] (6,1.8) rectangle (18.2,0.2);
\node[white] at (12.1,1) {\Huge{\textsc{Further Calculus}}};
\node at (3,1) {\Large{\textsc{Chapter 16}}};
\end{tikzpicture}
\end{center}

\chapterfont{\color{white}}

\chapter*{Answers}\label{answers}
\addcontentsline{toc}{chapter}{Answers}

\vspace{-10.5cm}

\begin{center}
    \begin{tikzpicture}
        \draw[white] (10,5) circle (0.01);
        \draw[black!50,fill=black!50] (0,0) rectangle (18.2,0.2);
        \draw[black!50,fill=black!50] (6,1.8) rectangle (18.2,0.2);
        \node[white] at (12.1,1) {\Huge{\textsc{Answers}}};
        \node[white] at (4,1) {\Large{\textsc{Chapter 13}}};
    \end{tikzpicture}
\end{center}

\chapterfont{\color{white}}

\chapter*{Challenge Problems}\label{challenge-problems}
\addcontentsline{toc}{chapter}{Challenge Problems}

\vspace{-10.5cm}

\begin{center}
    \begin{tikzpicture}
        \draw[white] (10,5) circle (0.01);
        \draw[black!50,fill=black!50] (0,0) rectangle (18.2,0.2);
        \draw[black!50,fill=black!50] (6,1.8) rectangle (18.2,0.2);
        \node[white] at (12.1,1) {\Huge{\textsc{Challenge Problems}}};
        \node[white] at (4,1) {\Large{\textsc{Chapter 13}}};
    \end{tikzpicture}
\end{center}

The following problems \textbf{do not} represent the kind of question expected to feature in a Higher Mathematics exam, either in the way they are presented or the level of difficulty. Instead, they aim to encourage a flexible approach towards problem-solving and an understanding that the skills covered in the course have applications beyond those featured in any typical exam. \emph{Some questions may be solveable without using the skills covered in this chapter, and some questions may be unreleated to this chapter.}

\bibliography{book.bib,packages.bib}

\end{document}
